\documentclass[10pt,twoside,letterpaper]{book}
\usepackage [english,spanish,activeacute]{babel}
\usepackage[latin1]{inputenc}
\usepackage{pstricks,pst-tree,pst-node}
\usepackage{synttree}
\usepackage{amsfonts,amsmath,amsthm,amssymb}
\usepackage[dvips]{graphicx}
\usepackage{enumerate,multicol,array}
\usepackage{color}
\usepackage{pifont}
\usepackage{makeidx}
\usepackage[T1]{fontenc}  %Este paquete permite que en el separasilabas podamos incluir palabra que tengan acentos castellanos
\usepackage{fancyhdr}
\usepackage[centerlast,small,bf]{caption2}
\usepackage{pst-all}
%%%%%%% Definici� del color gris %%%%%%%%%%%%%%%%%%%%
\definecolor{grisA}{cmyk}{0.0,0.0,0.0,0.8}
%%%%%%%%  Margenes %%%%%%%%%%%%%%%%%%%%%%%%%%%%%%%%%%%
\usepackage[body={330pt,523pt}]{geometry}
\setlength{\headsep}{2.5ex}%
\setlength{\footskip}{1.0ex}%
\setlength{\parindent}{0pt}    %%no indenter
\setlength{\topmargin}{10ex} %corre hacia arriba con -
\setlength{\parskip}{1.5ex plus 0.7ex minus 0.6ex}%%separaci� de p�rafos
\linespread{0.925}               %%interlineado
\renewcommand{\headrulewidth}{0.25pt} %grosor de la linea del encabezado
\renewcommand{\footrulewidth}{0.0pt}  %grosor de la linea al pie de pagina

%%%%%%%% redefiniciones utilizadas %%%%%%%%%%%%%%%%%%%%%%%%%
\newcommand{\mano}{{\Large\ding{42}}}
\newcommand{\flecha}{\xrightarrow{\ \ }}
\newcommand{\igual}{\mbox{\large$\boldsymbol =$}}

% mis comandos

\newcommand{\Cmd}[1]   {\texttt{#1}}
\newcommand{\Pname}[1] {\begin{Large}\texttt{#1}\end{Large}}
\newcommand{\Gui}[1]   {\begin{small}\textbf{#1}\end{small}}
\large
\title{Sistem\'atica filogen\'etica\\Introducci\'on a la pr\'actica}
\author{J. Salvador Arias,\\Ivonne J. Garz\'on-Ordu\~na y\\Daniel Rafael Miranda-Esquivel}
\date{Marzo, 2006. Ver 1.0}



\sloppy %
\makeindex %

 %%\includeonly{cap_2}
\begin{document}

 \thispagestyle{empty}
  \null
  \vspace{60mm}
  \begin{center}
  \Huge{\textbf{Sistem\'atica filogen\'etica\\Introducci\'on a la pr\'actica}}
  \end{center}
  \vfill
 \pagebreak

 \thispagestyle{empty}
  \null
  \vspace{10mm}
  \begin{center}
  \Large{\textbf{J. Salvador Arias,\\Ivonne J. Garz\'on-Ordu\~na y\\Daniel Rafael Miranda-Esquivel}}
  \end{center}
  \vspace{40mm}
  \begin{center}
  \Huge{\textbf{\\Marzo, 2006. Ver 1.0}}
  \end{center}
  \vspace{50mm}
 \pagebreak
 \thispagestyle{empty}
\frontmatter
 \thispagestyle{empty}
 \null
  \vfill
  \begin{center}{\Large
   \textit{A Jah, Jha, y Rha\\
   a Chango\\
   a Yemaya}}
  \end{center}
  \vfill
 \pagebreak
   \thispagestyle{empty}

\setcounter{page}{0}
 \pagestyle{fancyplain}
  \lhead[\fancyplain{}{\thepage}]
    {\fancyplain{}{Contenido}}%
  \rhead[\fancyplain{}{Contenido}]%
    {\fancyplain{}{\thepage}}
  \cfoot[]{} %
 \tableofcontents

\thispagestyle{empty}
\pagestyle{fancy}
\fancyhead{}
\fancyfoot{}
\renewcommand{\chaptername}{}
\fancyhead[LE]{ \textbf{Sistem\'atica Filogen\'etica. Introducci\'on}}
\fancyhead[RO]{ \textbf{Introducci\'on a la pr\'actica.}}
\fancyfoot[CE,CO]{\thepage}
\chapter{Introducci'on general}
Este libro ha sido pensado como uno de los tantos soportes posibles para las clases de sistem'atica de nivel b'asico y avanzado, adem'as de servir de repaso a conceptos te'oricos generales \textit{pero} sobre la base emp'irica, y no como \textbf{reemplazo} de los 
libros b'asicos o avanzados sobre an'alisis filogen'etico.
\section*{?`C'omo est'a estructurado este manual?}
El libro consta de 12 secciones que van desde manejo de caracteres, pasando por editores de matrices y 'arboles a b'usquedas tanto para an'alisis de parsimonia y m'axima verosimilitud (ML), como para an'alisis bayesiano. En todos los casos se presentan las t'ecnicas, la metodolog'ia a seguir, los programas de c'omputo a usar (y sus \Cmd{comandos}), adem'as de una serie de preguntas sobre la pr'actica o en general sobre la t'ecnica. La literatura recomendada es una sugerencia de lecturas, desde el punto de vista de los autores, cr'iticas, pero obviamente no cubre todos los art'iculos posibles; exploraciones constantes de revistas como \textit{Cladistics}, \textit{Systematic Biology}, \textit{Zoologica Scripta} o \textit{Molecular Phylogenetics and Evolution} y similares, actualizar'ian las perspectivas aqu'i presentadas. Al final se presenta un cap'itulo que trata sobre los programas usados, que se espera funcione como una guia r'apida para el uso de los mismos pero que no reemplaza al manual.\\
A lo largo del libro se utiliza una tipograf'ia consistente para indicar el nombre de los programas (v.g., \Pname{TNT}, \Pname{Component}), las instrucciones que deben ser escritas en los programas de l'inea de comandos (v.g. , \Cmd{mult=replic 10;}) y las instrucciones que se acceden mediante un men'u o un cuadro de dialogo en los programas de interfaz gr'afica (v.g., \Gui{Analize/heuristics}).\\
El orden de los cap'itulos obedece a la estrategia de ense~nanza de DRME. Obviamente el manual puede ser seguido en otros 'ordenes, por ejemplo todo sobre caracteres, excluyendo alineamiento, seguido de b'usquedas incluida la b'usqueda del modelo, consensos y por 'ultimo soporte y al final discutir alineamiento.\\ 
Usted podr'a encontrar material adicional, los datos, algunos macros para los distintos programas y dem'as chismes en el sitio web del laboratorio de Sistem\'atica \& Biogeograf\'ia (LSB) de la UIS, en la direcci'on: http://tux.uis.edu.co/labsist.
% \href{http://tux.uis.edu.co/labsist/docencia}.
\section*{P'ublico objeto}
Se espera que el usuario de este manual tenga conocimientos b'asicos, o que est'e tomando un curso formal de sistem'atica filogen'etica a nivel de pre o posgrado. No se esperan caracter'isticas especiales en cuanto a dominio de computadores, pero el manejo de un editor de texto que permita grabar archivos sin formato es \emph{muy} recomendable, adem'as de manejar el entorno de l'inea de comandos, el cual es usado en una gran cantidad de programas. Los usuarios pueden trabajar en cualquier plataforma desde Linux a MacOSX, pasando por Windows; en general se citan los programas apropiados para cada plataforma.
\section*{C'omo se gest'o este libro}
Es necesario aclarar que el orden de los autores sigue el orden ''natural'' del alfabeto. El piloto se construy'o sobre las pr'acticas de sistem'atica b'asica para el programa de biolog'ia de la UIS a cargo de DRME, donde los dos primeros autores actuaron como auxiliares docentes en los cursos de 2003, 2004, o como profesor a cargo de la parte pr\'actica (JSA) en 2006; adicionalmente sobre los cursos de posgrado de DRME en UdeA. Aunque la parte inicial se hizo sobre \textit{software} propietario, la configuraci'on del documento final se hizo sobre \LaTeX, a expensas de los dolores de cabeza de los dos primeros autores. La multiplicidad de las plataformas se debe a la multiplicidad de autores y a sus gustos o disgustos sobre estas. Windows es v'alida para los tres, MacOSX para IJGO y Linux para DRME.\\
A este manual le espera un largo y constante proceso de edici'on, por lo que sugerencias y recomendaciones ser'an siempre bienvenidas. Dado que la sistem'atica filogen'etica incluye el estudio de la distribuci'on de los organismos, un manual como este para biogeograf'ia est'a en construcci'on. Visitando nuestra p'agina web podr'a saber m'as acerca de 'el.
\section*{Agradecimientos}
A la Universidad Industrial de Santander, en particular a la Escuela de Biolog\'ia. A los estudiantes del programa de pregrado de la escuela de Biolog'ia de la UIS (Universidad Industrial de Santander) y del posgrado de la UdeA (Universidad de Antioquia). A. Brower, y C. Penz nos ayudaron con sus sugerencias y/o comentarios al escrito y C. Mayorga realiz'o una profunda revisi'on de estilo. R. Page gentilmente nos permiti\'o reproducir la figura de los camin'aculos.

\mainmatter
%%%%%%%%%%%%%%%%%%%%%%%%%%%%%%%%%%%%%%%%%%%%%%%%%%%%%%%%%%%%%%%%%%%%
%\pagestyle{fancyplain}
%%\addtolength{\headwidth}{0.0pt}%\marginparwidth
\renewcommand{\chaptermark}[1]{\markboth{\thechapter{}. #1}{}}
\renewcommand{\sectionmark}[1]{\markright{\thesection{}. #1}}
\lhead[\fancyplain{}{\thepage}]
    {\fancyplain{}{\rightmark}}%\bfseries
\rhead[\fancyplain{}{\leftmark}]%\bfseries
    {\fancyplain{}{\thepage}}
\renewcommand{\baselinestretch}{0.75}

%garenasd@uis.edu.co

\mainmatter
% 
\input{formato.tex}
\setcounter{page}{0}
 \thispagestyle{empty}
  \null
\chapter{Selecci\'on de caracteres}
\section*{Introducci\'on}
\index{Caracter!Homolog\'ia}
\index{Caracter!Definici\'on}

En un an\'alisis filogen\'etico clad\'istico es necesario identificar y usar caracteres hom\'ologos, de hecho, la importancia de los caracteres es tal que Hennig (1968) se refer\'ia a los espec\'imenes estudiados como semaforontes (\textbf{o los que llevan caracteres}). A\'un a pesar del papel central del concepto de car\'acter, la definici\'on no es tan f\'acil.


Richards se\~nala que el t\'ermino 
\textbf{car\'acter} no est\'a bien definido (Richards, 2003) y que su 
reconocimiento depende en gran parte del entrenamiento del tax\'onomo 
(Richards, 2002). 
Es por eso que la mayor\'ia de las ocasiones se 
enfatiza en la importancia de un \textbf{an\'alisis de caracteres} 
profundo y concienzudo (v.g., Hennig, 1968; Neff, 1986; Rieppel \& 
Kearny, 2001)\footnote{Este an\'alisis de caracteres tambi\'en implica revisar las afirmaciones cuatitativas o cualitativas hechas, ver 
Wiens 2001 
% Syst Biol (2001) 
% Character Analysis in Morphological Phylogenetics: Problems and Solutions
% 50 (5): 689-699.doi: 10.1080/106351501753328811
}.
\index{Caracter!Codificaci\'on} 

Las ideas de car\'acter y homolog\'ia usadas aqu\'i 
son independientes del tipo de caracteres usados, pero es mucho m\'as 
sencillo presentar la discusi\'on en t\'erminos de caracteres 
morfol\'ogicos. M\'as adelante se tratar\'a directamente el tema de los 
caracteres moleculares. 

Aunque no existe una f\'ormula m\'agica para reconocer caracteres hom\'ologos, 
la principal idea es la similitud ''especial'', basada en 
la definici\'on de homolog\'ia como \textbf {similar por ancestr\'ia 
com\'un} (Rieppel \& Kearny, 2001; \textit {contra} Kluge 2002, 
Grant \& Kluge, 2004). Por similitud no debe entenderse solo el 
parecido general, sino que existe una correlaci\'on estructural y de 
posici\'on (la similitud y conjunci\'on en el sentido de Patterson, 
1981) del car\'acter en los taxa comparados.\index{Caracter!Test 
de Patterson} 

Por ejemplo, los brazos humanos, las alas de los murci\'elagos y las 
patas de los caballos presentan una estructura \'osea muy similar, 
con diversos huesos colocados en las mismas posiciones, y en general 
el miembro completo est\'a en la misma posici\'on con respecto al 
cuerpo en los tres taxa, aunque sus formas son muy diferentes. 


Otro factor importante a tener en cuenta son los ''grados'' o 
niveles de generalidad de homolog\'ia. Es decir, que una estructura 
es hom\'ologa con otra en un cierto grado, pero no comparable con 
ella en otro. Por ejemplo, artr\'opodos y vertebrados comparten la 
cefalizaci\'on y el tener miembros pareados. Comparar directamente 
esas estructuras no es apropiado, aunque es posible que sean 
hom\'ologas a nivel molecular, donde los mismos genes controlan el 
desarrollo de estas estructuras, pero la larga historia 
independiente de cada linaje hace imposible una comparaci\'on de las 
estructuras como homolog\'ias, aunque no prohibe una comparaci\'on 
funcional. 

Es importante recalcar que cuando alguien compara los 
caracteres de diferentes taxa, hace una inferencia fuerte sobre los 
caracteres: se trata del mismo car\'acter en los taxa; es decir, da 
una identidad hist\'orica al car\'acter al hacerlo el mismo car\'acter 
por homolog\'ia, aunque est\'en modificados los distintos 
estados (Hennig, 1968; Kluge, 2002). Esto es lo que le da soporte al 
uso de la congruencia de caracteres para descubrir las sinapomorf\'
ias que le dan identidad hist\'orica a los grupos supraespec\'ificos.

 
La selecci\'on y observaci\'on de caracteres, no solo morfol\'ogicos, es un trabajo que s\'olo se aprende mediante el trabajo continuo con los datos derivados de los espec\'imenes en el laboratorio y con una lectura cr\'itica de la literatura sobre el grupo. Es importante que usted pueda reconocer cu\'ando los caracteres usados en una descripci\'on, una clave o un an\'alisis filogen\'etico cumplen o no con el requisito de la identidad hist\'orica.


Algunos libros de texto  sistem\'atica, presentan distintos criterios de reconocimiento de caracteres (por ejemplo Hennig, 1968; Wiley \&  Lieberman 2011), aunque no es posible plantear reglas estrictas o situaciones de casos perfectos, eso no es un motivo para suponer que los caracteres morfol\'ogicos carecen de base conceptual y por ello deben ser excluidos del an\'alisis (\textit{contra} Scotland et al., 2003).

Existen algunos criterios b\'asicos aplicables en general a los caracteres: para cada car\'acter se reconocen diversos estados alternativos de ese car\'acter, los cuales son llamados estados de car\'acter o simplemente estados. Es posible que un car\'acter no sea aplicable a todos los organismos de estudio, pero dentro de los organismos donde es aplicable, los estados de un car\'acter deben dividir el universo en al menos dos grupos mutuamente excluyentes, cada uno de ellos reconocible por un estado de caracter.


Dado que se tratan los caracteres como identidades hist\'oricas, se debe ser cuidadoso con \textbf{NO} crear estados de car\'acter que sirvan como ''cajas de basura''. Por ejemplo, en el car\'acter ''color: (a) rojo; (b) otro color'', el estado (b) es ambiguo y puede contener individuos de diferentes colores, cuya \'unica (falsa) similitud sea que no poseen el color rojo, pero no existe una identidad hist\'orica defendible, es decir el NO color rojo se ha generado desde m\'ultiples ancestros.


Todos los estados del car\'acter deben representar una unidad hist\'orica, independientemente de si el estado es el apom\'orfico o el 
plesiom\'orfico. 
Siempre que encuentre un car\'acter donde uno de los estados est\'a definido como negaci\'on (incluidas las ausencias), 
debe revisar cuidadosamente que el estado ''negativo'' indique claramente una unidad.


Cuando los caracteres solo tienen dos estados, no es necesaria ninguna suposici\'on sobre la ''direcci\'on del cambio''; 
sin embargo, cuando hay m\'as de dos estados, la direcci\'on es una pregunta importante. 
Algunos caracteres simplemente no dan posibilidad de escoger una direcci\'on particular (por ejemplo, las bases nucleot\'idicas), 
por lo cual son caracteres no ordenados. 
Bajo una posici\'on,  uno puede asignar un posible orden a los caracteres que se leen como diversos grados de homolog\'ia, 
sin asumir nada sobre el proceso evolutivo, 
por ejemplo, el caracter: 


\begin{small}
Pilosidad en la pata: 
\begin{enumerate}[start=0]
\item solo en el extremo anterior
\item hasta la mitad
\item todal a superficie
\item toda la pata y con presencia de pelos m\'as largos en el extremo posterior
\end{enumerate}	
\end{small}


podr\'ia ordenarse desde 0 hasta 3 (hacerlo aditivo) para reflejar que existe un mayor grado de homolog\'ia entre, por ejemplo, el estado 2 y 3, que 1 y 3. 
La otra posici\'on no privilegiar\'a ning\'un cambio con respecto a otro y la direcci\'on se asignar\'ia por la congruencia con otros caracteres (el car\'acter es no aditivo).


En otros casos, la identidad hist\'orica es dif\'icil de apreciar; por ejemplo, ''transparentable con KOH: (a) s\'i, (b) no'', suponiendo que el estado (b) no es problem\'atico, el car\'acter como un todo parece ser m\'as un artefacto de laboratorio que una propiedad heredable del organismo. En esa misma l\'inea est\'an caracteres como ''n\'umero de componentes en PCA: (a) uno; (b) dos''.


Al tratarse los estados de caracter como alternativas, un mismo organismo no puede poseer dos estados o m\'as estados del mismo car\'acter: para nuestra visi\'on de homolog\'ia de las alas y los miembros anteriores, no pueden existir \'angeles o centauros, con dos versiones diferentes del miembro anterior en el mismo organismo. Este es el criterio de ''conjunci\'on'' de Patterson. sin embargo, hay que tener en cuenta que existen homolog\'ias seriadas que \textbf{contradicen} esta idea, no solo en organismos segmentados como an\'elidos y artr\'opodos, sino en otros donde la segmentaci\'on no es clara (e.g., hojas de las plantas, pelos de los mam\'iferos, etc.). El criterio de conjunci\'on de Patterson es un llamado al reconocimiento de las estructuras que se usan como caracteres en el an\'alisis filogen\'etico.

%\section{Materiales}

\section*{T\'ecnicas}


Use el dibujo al final de la pr\'actica el c\'ual corresponde a la figura 1 en Sokal (1983). 
Los ''organismos'' son camin\'aculos, seres hipot\'eticos creados por J. Camin para estudios sobre clasificaciones en fen\'etica.\footnote{Copyright, Society of Systematic Biologists, se reproduce con permiso} A partir de los dibujos:


%\section{M\'etodos}


\begin{enumerate}
	\item Elabore una peque\~na gu\'ia morfol\'ogica de los organismos, de modo que pueda diferenciar las diferentes estructuras en los diferentes organismos (en buen romance significa poner nombres a sus organismos y ligar tales nombres a estructuras f\'acilmente reconocibles; el ejercicio m\'as cercano en un an\'alisis para separar por \textbf{morfos} sus organismos).

	\item Elabore un listado de los diferentes caracteres de los organismos, reconozca al menos 5 caracteres \'unicos y 5 compartidos. Anote en una hoja separada los organismos y sus estados.
	
	\item Intercambie su listado (acompa\~nado de la gu\'ia morfol\'ogica y los nombres de los organismos) con un compa\~nero: 
	
	\begin{enumerate}
		\item Trate de reconocer los caracteres que su compa\~nero describi\'o, y anote c\'uales organismos poseen esos caracteres.
		\item En caso de ser necesario, re-escriba los caracteres de su compa\~nero e indique la raz\'on de sus cambios.
	\end{enumerate}
	
	\item Compare la lista de caracteres por organismos que usted 
	realiz\'o con la que su compa\~nero re-escribio.

\end{enumerate}

\section*{Preguntas}

\subsection*{Pr\'actica}


\begin{enumerate}
	\item Haga un ejercicio cr\'itico de los caracteres de sus compa\~neros:
	 
	\begin{enumerate}
		\item ?`est\'an bien definidos?	
		\item ?`hay impl\'icita una transformaci\'on?
		\item ?`usaron de la misma forma la similaridad?
	\end{enumerate}

	\item Trate de localizar los motivos de sus aciertos y fallas cuando trabaj\'o con los caracteres de su compa\~nero. ?`Son esos motivos consistentes con la cr\'itica a los caracteres?

	\item Eval\'ue la cr\'itica de su compa\'nero a sus caracteres: ?`es posible reescribir los caracteres para mejorarlos? Si as\'i lo cree, reescr\'ibalos; en caso contrario, d\'e argumentos para desechar el car\'acter o para mantener su definici\'on actual.
\end{enumerate}


\subsection*{Generales}

\begin{enumerate}
	\item Use la literatura de la que disponga sobre un grupo en particular, trate de evaluar los caracteres presentes y h\'agase preguntas similares a las que realiz\'o cuando examin\'o los caracteres de sus compa\~neros.\\

	\item Si es posible, trate de usar y contrastar un an\'alisis filogen\'etico con una descripci\'on tradicional del mismo grupo.\\

	\item Pleijel (1995) argument\'o que los \'unicos caracteres v\'alidos son las presencias, por lo que todos los caracteres son una enumeraci\'on  de presencia/ausencia (no hay \textbf{estados de car\'acter}), trate de identificar las posibles ventajas y falencias de esa aproximaci\'on.\\

	\item Lea el art\'iculo de Goloboff et al., 2006 e identifique la forma como se manejan los caracteres cuantitat\'ivos. ?`usan un esquema similar al usado para los caracteres discretos?
\end{enumerate}



%\newpage
\section*{Literatura recomendada}

%
% revisar el esquema de citacion en esta sección
%

de Pinna (1991) [Ofrece una visi\'on a la formulaci\'on de los caracteres].

Kluge, 2003 [Una cr\'itica al uso de la \textbf{similitud} como criterio de selecci\'on de caracteres].

Neff, 1986 [Una normalizaci\'on del an\'alisis de caracteres].

Patterson, 1982 [La \textbf{verdadera prueba de homolog\'ia}].

Platnick, 1979 [Un art\'iculo cl\'asico sobre la jerarqu\'ia -cladismo de patr\'on-, tanto de caracteres como de taxa].

Pleijel, 1995 [Una visi\'on al manejo de caracteres, diferente a la ofrecida aqu\'i].

Rieppel \& Kearny, 2002 [Una visi\'on cr\'itica y extensa de la selecci\'on de caracteres, con ejemplos emp\'iricos].

\begin{figure}
\centering
\includegraphics[scale=0.75]{./practica01/cami0.jpg}
% \resizebox{7in}{7in}{\includegraphics[]{camin.jpg}}
% \scalebox{0.80}[0.40]{\includegraphics[clip,scale=1]{cami.jpg}}
% \resizebox{5in}{9in}{\includegraphics[clip,scale=1]{2_p.jpg}}
% \includegraphics[bb= 600 920 100 0, scale=0.30]{portada.jpg}
% \includegraphics[bb= 100 200 100 0,scale=1]{cami.jpg}
% ,scale=0.5
\caption{Los Camin\'aculos, tomado de Sokal (1983) \\Copyright \textit{Society of Systematic Biologists}, se reproduce con permiso.}
% \vspace{-2mm}
\end{figure}


\chapter{Matrices de datos}

\section*{Introducci\'on} 
\index{Caracter!Codificaci\'on} 

Una de las diferencias m\'as importantes entre los trabajos taxon\'omicos con un enfoque \textbf{cl\'asico} y los an\'alisis filogen\'eticos es que estos \'ultimos incluyen expl\'icitamente una matriz de datos donde se puede evidenciar los caracteres examinados, y c\'omo fueron interpretados.\\

B\'asicamente, la matriz de datos es una lista tabulada de las observaciones de los caracteres en los distintos taxa. Para facilitar su lectura y su uso en programas de an\'alisis filogen\'etico, los caracteres y sus estados son codificados como n\'umeros o como letras. Una vez construida, la matriz es el punto de partida para la b\'usqueda de \'arboles y la manera m\'as sencilla con la que otros investigadores pueden recuperar la informaci\'on recopilada durante el \textbf{an\'alisis de caracteres}\footnote{si desea puede ver nuestra visi\'on ampliada en Miranda et al., 2004.}.\\

Tal y como se hizo en la pr\'actica sobre selecci\'on de caracteres, lo que se busca es que el investigador sea consistente en la codificaci\'on de los caracteres, lo cual es importante en el manejo de caracteres inaplicables y no observados. Algunos autores los codifican de diferentes formas en la matriz (usualmente, con \textbf {-} y \textbf {?}) para facilitar la recuperaci\'on de informaci\'on (puede encontrar una discusi\'on m\'as completa en Strong \& Lipscomb, 1999),  otros autores, y en general los programas, no reconocen diferencia entre unos y otros, mientras que programas como \Pname{TNT} o \Pname{Poy} pueden reconocer como un quinto caracter los eventos de inserci\'on -perdida o gaps (Giribet \& Wheeler, 1999).\\
% Molecular Phylogenetics and Evolution
%Vol. 13, No. 1, October, pp. 132–143, 1999
%Article ID mpev.1999.0643, available online
%

Cuando se usan caracteres con m\'ultiples estados, es necesario clarificar cu\'ales fueron usados como aditivos y cu\'ales como no aditivos, o en el caso de haber sido recodificados, como se hizo esa codificaci\'on. Si se han codificado como aditivos, se debe indicar el ?`por qu\'e? de tal codificaci\'on e incluir los argumentos que muestren que los estados de car\'acter se hallan anidados entre s\'i, la aditividad puede generar al estrucutra en la topolog\'ia fianl, sin que haya relamente transformaciones qeu soporten los nodos.\\



\section*{T\'ecnicas}

Existen diferentes programas para manipular matrices de datos y \'arboles. Algunos de ellos permiten la interacci\'on matriz-\'arbol (
\Pname{WinClada}\footnote{\url{www.cladistics.com}}, 
\Pname{Mesquite}\footnote{\url{www.mesquite.org}}, 
\Pname{MacClade}\footnote{el programa no corre en versiones de MacOS X superiores a 10.6 (desde Lion en adelante)}, 
\Pname{R}\footnote{\Pname{R} es uno de los programas para an\'alis estad\'istico m\'as versatiles del momento, adem\'as de ser gratuito permite la implementaci\'on de m\'ultiples procesos de \'alculo, de tal manera que es posible realizar distintos tipos de operaciones y c\'alculos desde estadistica b\'asica hasta an\'alisis en evoluci\'on; sirve de plataforma para \Pname{ape} %faltan citas:
(Paradis et al., 2008), un m\'odulo que permite distintas acciones con matrices y \'arboles 
\url{cran.r-project.org}}  o 
\Pname{Seaview}\footnote{\url{doua.prabi.fr/software/seaview}}). En su mayor\'ia estos programas sirven de plataforma para manejar programas que realizan el an\'alisis filogen\'etico como tal. Muchos de los programas que manejan matrices est\'an dise\~nados b\'asicamente para alg\'un tipo particular de datos (por ejemplo, ADN) y aquellos que tienen un marco amplio se quedan cortos para manejar cierta clase de informaci\'on (por ejemplo, no interpretan la codificaci\'on IUPAC para polimorfismos de ADN).

Los programas pueden leer uno o varios formatos de archivos, pero solo algunos programas como \Pname{Mesquite} leen y escriben todos los formatos de datos; es importante revisar la compatibilidad de los programas para el manejo de archivos. En la mayor\'ia de ellos es posible exportar entre los diferentes tipos de datos, o por lo menos entre los m\'as usados. En general, la mayor parte de los programas trabaja bien con el formato NEXUS (Maddison, et al., 1997), 
%Maddison DR, Swofford DL, Maddison WP (1997). 
%"NEXUS: An extensible file format for systematic information". 
%Systematic Biology 46 (4): 590–621. doi:10.1093/sysbio/46.4.590. PMID 11975335.
tanto para exportar como para importar, aunque el formato es en ocasiones muy diferente y algunos programas pueden no identificarlo correctamente. El otro formato importante es el de Hennig86 o NONA, pero en muchos programas, especialmente moleculares, su uso no est\'a implementado.

Revise siempre la documentaci\'on del programa que desea usar y as\'i podr\'a estar seguro si el programa que va a utilizar cumple con los requisitos que usted necesita y cual es el formato de las matrices.

Existen listas de programas para an\'alisis filogen\'etico que pueden ser consultadas en internet, por ejemplo:


\url{http://evolution.genetics.washington.edu/phylip/software.html}
 
o en 

\url{http://taxonomy.zoology.gla.ac.uk/software/}

\section*{T\'ecnicas}

Abra el archivo de datos morfol\'ogicos para vertebrados \Datos{datos.vertebrados.xls} con un programa para hojas de c\'alculo y manuipule las matrices con \Pname{WinClada}, \Pname{Mesquite} o \Pname{TNT} para Windows:

\begin{enumerate}
	\item Anote cu\'ales caracteres son multiestado y cu\'ales son binarios.

	\item Identifique si hay o no caracteres aditivos.

	\item A partir de los datos, construya una matriz nueva (en \Pname{TNT} para Windows use el menu \Gui{Data - edit data}). 

	\item Dado que los programas no cuentan con opciones de salvado autom\'atico, peri\'odicamente salve la matriz.

	\item Nomine los terminales y los caracteres y sus estados. Explore diferentes formas de llevar esa tarea a cabo. 

	\item Suponga que algunos autores consideran que las plumas son escamas modificadas. Aceptando esa informaci\'on, recodifique la matriz y determine si el car\'acter es aditivo o no.

	\item Seleccione el \'ultimo car\'acter y col\'oquelo al principio de la matriz. 

	\item Introduzca un car\'acter nuevo en la posici\'on 4.

	\item Exporte la matriz en otros formatos, \Fname{NEXUS} si est\'a usando \Pname{WinClada}, \Fname{Nona} si est\'a usando \Pname{Mesquite}.

	\item Verifique la compatibilidad de los datos, abriendo la matriz en el programa correspondiente.

	\item Revise con un editor de texto los archivos que usted cre\'o, trate de identificar cu\'ales son las partes claves del formato\footnote{Aunque en general no se usan los editores de texto, este paso es cr\'itico para despu\'es ser capaz de rastrear los problemas que pueden tener las matrices de datos.}.

	\item Abra una de las matrices moleculares en cada programa y con un editor de texto y trate de identificar en qu\'e se diferencia de la matriz morfol\'ogica.

	\item en R:
	\begin{itemize}
		\item Instale en su ordenador la versi\'on m\'as reciente de R ($\ge$ 3.00-11). 

		\item Carge las bibliotecas \Rlib{ape} y \Rlib{phangorn}\footnote{La biblioteca \Rlib{phangorn} est\'a diseñada para an\'alisis filogen\'eticos que tiene como objetivo estimar \'arboles y redes,  utilizando diferentes m\'etodos como m\'axima verosimilitud,  parsimonia, distancia y conjugaci\'on de Handamard. Requiere las bibliotecas $"$ape$"$ y $"$rgl$"$,  y puede ser descargado desde \url {http://cran.r-project.org/web/packages/phangorn/index.html}}: 
		\Cmd{library(ape)}
		\Cmd{library(phangorn)}

		\item En caso de no tenerlas disponibles primero bajelas con la instrucci\'on:
		\Cmd{install.packages(c($"$ape$"$,$"$phangorn$"$), dependencies=T)}\footnote{La funci\'on \Rfunc{install.packages($"$Nombre\_paquete$"$)} permite hacer la descarga de los paquetes directamente del repositorio desde el entorno de R. Tambi\'en es posible hacer la descarga directamente de la pagina web e instalarlo desde cualquier directorio en el ordenador dando directamente la ruta a dicho sitio: 
		\Cmd{install.packages ($"$../usuario/R/Packages/Paquete.tar.gz$"$)}
		
		Tenga en cuenta que de este modo deberá descargar e instalar todas las dependencias requeridas por el paquete de manera independiente. El comando library() le permitir\'a cargar el paquete en el entorno de R para poder empezar a trabajar con todas sus utilidades.}

		\item Abra la matriz de datos en formato de texto simple y asignela a un objeto de R: 
		\Cmd{ datos $<-$ read.table($"$matriz.txt$"$)}

		\item Revise los nombres de las variables en la matriz con
		\Cmd{names(datos)}

		\item Revise los nombres de las terminales en la matriz con
		\Cmd{row.names(datos)}

		\item Abra la matriz de datos en formato \Fname{phylip} y asignela a un objeto de R: 
		\Cmd{datos.Phylip $<-$ read.phyDat($"$DNA1.phy$"$, format=$"$phylip$"$, type=$"$DNA$"$)} 

		\item Escriba en formato  \Fname{Nexus} la matriz leida anteriomente:
		\Cmd{write.nexus.data(datos.Phylip,file=$"$NuevaDNA1.nex$"$)}

		\item Abra la matriz escrita con \Pname{Winclada} o \Pname{Mesquite} y revise la conversi\'on.
	\end{itemize}

	\item En el directorio de datos existen dos matrices con distintos problemas \Datos{problema1.txt} y \Datos{problema2.txt}, intente abrir las matrices, busque  y corrija el error.\footnote{tradicionalmente, las terminaciones de los archivos son .ss en \Pname{WinClada}, .nex con \Pname{Mesquite} o \Pname{MacClade}, pero \textbf{debe} recordar que la terminaci\'on del archivo no es necesariamente el formato.}

\end{enumerate} 


Dependiendo de la plataforma que trabaje, usted tiene disponible distintos programas. \Pname{Mesquite} y  \Pname{R} son programas gratuitos y v\'alidos para todas las plataformas pero en general las b\'usquedas no son eficientes, aunque le permiten trabajar con varios tipos de datos e inreractuar directa o indirectamente con programas como \Pname{PAUP},\Pname{TNT} o \Pname{PhyML}. Sobre Windows usted cuenta con \Pname{WinClada} que funciona tanto en modo de manejo de edici\'on de matrices (\Gui {windada}) como en modo de edici\'on y manipulaci\'on de \'arboles; adem\'as, le permite hacer b\'usquedas con NONA, tambi\'en puede usar la versi\'on de Windows de \Pname{TNT} que tiene interface gr\'afica. Si usa Mac una opci\'on es \Pname{McClade}, el cual funciona como \Pname{Mesquite} pero es m\'as veloz y eficiente, aunque es muy factible que no lo pueda usar con la versi\'on actual de Mac OS X.


Revise la secci\'on de Programas de c\'omputo para ver los comandos que utilizar\'ia con un programa distinto a \Pname{Winclada/NONA}. En todos los casos familiar\'icese con los men\'us e instrucciones para abrir/cerrar y editar tanto matrices como \'arboles, tenga en cuenta que los manuales de los programas traen informaci\'on adicional, por lo tanto su lectura es una muy buena opci\'on.

\preguntaGral{
\begin{enumerate}

	\item Si desea transformar de un formato de matriz a otro usando exclusivamente editor de texto, ?`cu\'ales son los pasos a seguir? Ensaye la lista contruida con un ejemplo.

	\item Haga el listado de los aspectos comunes a todos los formatos de matrices.

	\item En el laboratorio anterior se insisti\'o en la claridad de los caracteres y sus estados. A la luz de los resultados obtenidos usando \'arboles y la matriz:

\begin{enumerate}

	\item ?`puede usted conectar la importancia del an\'alisis de caracteres con la forma como se interpretan los cladogramas?

	\item Revise su bibliograf\'ia, y de ser posible compare trabajos con y sin matriz expl\'icita. ?`Puede notar alguna diferencia?

	\item ?`Cree que es una ventaja incluir y publicar la matriz, o por el contrario es una desventaja?\\ 
\end{enumerate}


\end{enumerate}
}





\section*{Literatura recomendada}  

Maddison et al., 1997 [Una introducci\'on al formato NEXUS]. 

cita mesquite y sus bibliotecas

cita a NN? hennig? por que las matrices son importantes?
\chapter{'Arboles}
\section*{Introducci'on}
Una vez realizado un an'alisis filogen'etico, el resultado es un agrupamiento que tradicionalmente se dibuja como un 'arbol donde los taxa usados est'an como \textit{hojas} o \textit{terminales} del 'arbol. Las diferentes ramas se unen en \textit{nodos} o \textit{componentes}. As\'i, el agrupamiento ((Terminal 1, Terminal 2), Terminal 3) puede representarse con el \'arbol:\\


 \begin{center}
%    esta parte funciona en ps-tree pero con pdfetex no marcha
%\pstree[treemode=R]%
%{\TR{Ra\'iz}}
%{\TR{terminal1}\TR{terminal2{\TR{nodo3}}{\TR{terminal3.1}\TR{terminal3.2}}}}
%
\synttree[\textit{Ra'iz}[\textit{Nodo 1}[\textbf{Terminal 1}][\textbf{Terminal 2}]][\textbf{Terminal 3}]]

\end{center}

Es importante diferenciar entre el \textbf{contenido} del nodo, que son los terminales que forman el nodo (el nodo 1 tiene dos terminales: 1 y 2), y la \textbf{informaci'on} del nodo, que es c'omo est'an organizados esos terminales (Nelson, 1979). En general, para muchos autores un componente representa un grupo monofil'etico (i.e., definici'on \textit{topol'ogica} de monofilia).\\
En clad'istica, los 'arboles son tambi'en conocidos como \textbf{cladogramas} y el t'ermino es usado sin distinci'on. En clad'istica (pero no en otras metodolog'ias) los cladogramas deben tener los nodos soportados por transformaciones de caracteres. En caso de no tener transformaciones, el nodo no est'a soportado o es un nodo de longitud cero, es decir, es un artefacto de los programas usados para obtener los 'arboles (v'ease Coddington \& Scharff, 1994; Goloboff, 1998). El n'umero de transformaciones es un estad'istico importante que hay que tener en cuenta, en la mayor'ia de los casos se lo usa como criterio para seleccionar entre diferentes cladogramas, se escoge el que sugiera el  menor n\'umero de transformaciones.\\
En los m'etodos estad'isticos generalmente es importante la longitud de las ramas. Esta suele representarse como una probabilidad o frecuencia de transformaciones y suelen presentarse los 'arboles con una escala que muestra la longitud. Sol'o la dimensi'on que va de las ramas a los terminales tiene importancia en ese caso, el ''ancho'' del 'arbol solo es para acomodar los diferentes terminales usados. En los cladogramas, las dimensiones no tienen ning'un significado especial.
\section{T'ecnicas}
Existen diferentes programas para manipular y generar 'arboles; la mayor\'ia de ellos permiten interacci'on matriz-'arbol, mientras que otros solo utilizan 'arboles. En general, los programas de an'alisis filogen'etico permiten manipular 'arboles con una presentaci\'on gr'afica rudimentaria. Algunos programas permiten guardar informaci\'on adicional a la topolog\'ia en el mismo 'arbol como la longitud de la rama que conduce al nodo o terminal o una etiqueta. Los programas que pueden trabajar independientemente de la matriz suelen estar dise~nados para la impresi'on o exportaci'on gr'afica de los 'arboles, soportando cambios de topolog'ia, longitudes y etiquetas de las ramas, pero no transformaciones de caracteres.
\\
Una forma de expresar los 'arboles en formato de texto (por ejemplo para usar con programas o para el resumen de un art'iculo) es la notaci'on de par'entesis, donde  los par'entesis limitan los nodos, dependiendo de los autores o los programas, los grupos hermanos son separados por espacios, comas o s'imbolos de suma; por ejemplo (a (b c)), es equivalente a (a+(b+c)). Finalmente tenga en cuenta que en los 'arboles filogen'eticos el orden de los terminales, sin cambiar la topolog\'ia, no altera el contenido del 'arbol. Por ejemplo (a (b c)) es exactamente igual a ((c b) a) y a ((b c) a).
\\
\section{Materiales}
\noindent
Matriz de datos con 'arbol (datos.matriz.dat).\\
'Arbol simple (datos.tre).\\
'Arbol con longitud de ramas (datos.long.tre).
\section{M'etodos}
\noindent
\textbf{Tanto en \Pname{TreeView}} (usando solo el 'arbol) \textbf{como en \Pname{WinClada}, \Pname{TNT} (para Windows), \Pname{McClade} o en \Pname{Mesquite}}: (usando el archivo con la matriz incluida)\\
(1) Cambie la ra'iz del 'arbol.\\
(2) Cambie de posici'on nodos y terminales.\\
(3) Colapse alg'un nodo.\\
\textbf{En \Pname{Mesquite} y \Pname{TreeView}:}\\
(4) Cambie el orden de los terminales sin cambiar la topolog'ia (las relaciones entre terminales) del 'arbol.\\
(5) Coloque etiquetas en los nodos, y salve el 'arbol\\
(6) Revise ese archivo en un editor de texto para determinar c'omo se colocaron tanto las etiquetas como las longitudes.\\
(7) Abra el 'arbol con longitud de ramas y coloque el men'u \Gui{Tree-Phylogram}.
\subsection{Programas}
Dependiendo de la plataforma que trabaje, usted tiene disponible distintos programas. \Pname{Mesquite} es v'alido para todas las plataformas (y es gratuito) pero es muy lento si su m'aquina es lenta (requiere de la m'aquina virtual de Java). Sobre Windows usted cuenta con \Pname{WinClada} (requiere la matriz de datos), su interfaz de impresi'on requiere mucha pr'actica (no es intuitiva); tambi'en para Windows se ha desarrollado \Pname{TreeView}, pero este tiene el problema que los 'arboles est'an escalados al tama~no de la ventana, lo cual puede ser un problema si los 'arboles son muy grandes. Si usa Mac la opci'on obvia es \Pname{McClade}, el cual funciona como \Pname{Mesquite} pero m'as veloz y en algunos casos m'as eficiente. Otra alternativa, si usa Linux, pueden ser \Pname{NJPlot}. Para la impresi'on \textbf{final} del 'arbol para publicaci'on, una opci'on puede ser \Pname{Treegraph}\footnote{http://www.math.uni-bonn.de/people/jmueller/extra/treegraph}.
\subsection{Comandos}
En general, la manipulaci'on de cladogramas se activa con un men'u (en \Pname{WinClada} \Gui{Edit/Mouse}, en \Pname{TreeView} \Gui{Edit-Edit tree}), o est'a directamente como barra de herramientas en la ventana del 'arbol (\Pname{Mesquite}, \Pname{McClade}, Ventana de edici'on de \Pname{TreeView}). Las acciones se realizan al seleccionar con el rat'on la rama (\Pname{WinClada}) o arrastrando las ramas (para moverlas en \Pname{TreeView}). M'as que comandos, es importante manipular el rat'on.
\section{Preguntas}
\subsection{Pr'actica}
\noindent
(1) Dibuje el siguiente cladograma:\\ 
(Lampreas (Tiburones (Esturi'on Tele'osteos) (Celacanto (Peces-pulmonados (Anfibios (Mam'iferos (Tortugas (Lagartos (Cocodrilos Aves)))))))))\\
(2) Si la ra'iz esta colocada entre el Celacanto y los Peces-pulmonados, ?`c'omo es la topolog'ia resultante?\\
(3) Convierta la topolog'ia dibujada en (2) a notaci'on de par'entesis.\\
(4) Tome alg'un cladograma de la literatura y escr'ibalo en notaci'on de par'entesis (!`trate de no usar uno muy grande para evitar confusi'on!)
\subsection{Generales}
\noindent
Aparte de no tener un an'alisis expl\'icito, existe una gran diferencia entre los 'arboles filogen'eticos actuales y sus ''equivalentes'' usados por algunos tax'onomos (por ejemplo, Haeckel, Romer, etc.) ?`Es usted capaz de descubrirla? \textbf{Clave}: intente dibujar alguno de esos 'arboles al estilo actual.
\section{Literatura recomendada}
\noindent
Page \& Holmes, 1998 [El cap'itulo 2 esta dedicado a los 'arboles y presenta muy buenas ilustraciones. Tambi'en puede consultar la p'agina de docencia de Rod Page http://taxonomy.zoology.gla.ac.uk/teaching/index.html].\chapter{\'Arboles}
\section*{Introducci\'on}

Una vez finalizado un an\'alisis filogen\'etico, el resultado es un agrupamiento que tradicionalmente se dibuja como un \'arbol donde los taxa usados estan como \emph{hojas} o \emph{terminales} del \'arbol. Las diferentes ramas se unen en \emph{nodos} o \emph{componentes}. As\'i, el agrupamiento ((Terminal 1, Terminal 2), Terminal 3) puede representarse con el \'arbol:

\begin{center}
%
% http://tex.stackexchange.com/questions/2340/how-to-make-a-3-level-deep-tree-with-tikz
%
\begin{tikzpicture}[level distance=3cm,
  level 1/.style={sibling distance=9cm},
  level 2/.style={sibling distance=6cm}]
  \node {Ra\'iz}
    child {node {Terminal 3}}
    child {node {Nodo interno 1}
    	child {node {Terminal 1 = nodo (hoja)}}
      	child {node {Terminal 2}}
    };
\end{tikzpicture}
\end{center}

Es importante diferenciar entre el \textbf{contenido} del nodo, que son los terminales que forman el nodo (el nodo interno 1 tiene dos terminales: 1 y 2), y la \textbf{informaci\'on} del nodo, que es c\'omo est\'an organizados estos terminales en terminos de grupos o clados anidados (Nelson, 1979). En general, para muchos autores un componente representa un grupo monofil\'etico (i.e., definici\'on \emph{topol\'ogica} de monofilia).

En clad\'istica, los \'arboles son tambi\'en conocidos como \textbf{cladogramas} y el t\'ermino es usado sin distinci\'on. En clad\'istica (pero no en otras metodolog\'ias) los cladogramas deben tener transformaciones de caracteres en nodos (nodos soportados), en caso de no tener transformaciones, el nodo no est\'a soportado o es un nodo de longitud cero, es decir, es un artefacto de los programas usados para obtener los \'arboles (v\'ease Coddington \& Scharff, 1994; Goloboff, 1998). El n\'umero de transformaciones es un estad\'istico importante que hay que tener en cuenta y se usa como criterio para seleccionar entre diferentes cladogramas, se escoge la topolog\'ia que sugiere el  menor n\'umero de transformaciones. En los m\'etodos estad\'isticos generalmente es importante la longitud de las ramas, esta suele representarse como una probabilidad o frecuencia de transformaciones y suelen presentarse los \'arboles con una escala que muestra la longitud de ramas; s\'olo la dimensi\'on que va de las ramas a los terminales tiene importancia en ese caso, el ``ancho'' del \'arbol se usa para acomodar los diferentes terminales usados. En los cladogramas, las dimensiones no tienen ning\'un significado especial.


\section*{T\'ecnicas}

Una forma de expresar los \'arboles en formato de texto (por ejemplo para usar con programas o para el resumen de un art\'iculo) es la notaci\'on de par\'entesis, donde los par\'entesis limitan los nodos; dependiendo de los autores o los programas, los grupos hermanos son separados por espacios, comas o s\'imbolos de suma, por ejemplo (a (b c)), es equivalente a (a+(b+c)) y a (a, (b, c)), tenga en cuenta que en los \'arboles filogen\'eticos el orden de los terminales, sin cambiar la topolog\'ia, no altera el contenido del \'arbol, por ejemplo (a (b c)) es exactamente igual a ((c b) a) y a ((b c) a).

Existen diferentes programas para manipular y generar \'arboles; la mayor\'ia de ellos permiten interacci\'on matriz-\'arbol, mientras que otros solo utilizan \'arboles. En general, los programas de an\'alisis filogen\'etico permiten manipular \'arboles en un esquema gr\'afico rudimentario. Algunos programas permiten guardar informaci\'on adicional a la topolog\'ia en el mismo \'arbol como la longitud de la rama que conduce al nodo o terminal o una etiqueta. Los programas que pueden trabajar independientemente de la matriz suelen estar dise\~nados para la impresi\'on o exportaci\'on gr\'afica de los \'arboles, soportando cambios de topolog\'ia, longitudes y etiquetas de las ramas, pero no transformaciones de caracteres.

Dependiendo de la plataforma que trabaje, usted tiene disponible distintos programas: \Pname{Mesquite}, \Pname{R}\footnote{\url{cran.r-project.org}} y \Pname{Figtree}\footnote{\urlt{tree.bio.ed.ac.uk/software/figtree/}} son v\'alidos para todas las plataformas (adicionalmente son gratuitos). \Pname{Mesquite} es muy lento si su m\'aquina es lenta (requiere de la m\'aquina virtual de Java). Sobre Windows usted cuenta con \Pname{WinClada} (requiere la matriz de datos), su interfaz de impresi\'on requiere mucha pr\'actica (no es intuitiva). Otras alternativa, si usa Linux, pueden ser \Pname{NJPlot o GNUplot}. Para la impresi\'on final del \'arbol para publicaci\'on, una opci\'on puede ser \Pname{R}, que aunque es poco intuitivo al iniciar, logra resultados fianles de mejor calidad que otros programas.

En general, la manipulaci\'on de arboles se activa con un men'u (en \Pname{WinClada} \Gui{Edit/Mouse}, en \Pname{FigTree} directamente como barra de herramientas en la ventana; en \Pname{Mesquite} se usa el men\'u de herramientas de la ventana de edici'on de \Pname{TreeView}). Las acciones se realizan al seleccionar con el rat\'on la rama (\Pname{WinClada}) o arrastrando las ramas (para moverlas en \Pname{Figtree}). M'as que comandos, en esta pr\'actica es importante manipular el rat\'on.}}


\begin{enumerate}

	\item Desde \Pname{WinClada+NONA}, \Pname{Mesquite} o \Pname{TNT}
	\begin{enumerate}

		\item Revise los archivos de datos con editor de texto y con le programa seleccionado abra una matriz de datos que contenga tanto datos como \'arboles.

		\item Establezca la forma de obtener informaci\'on sobre los \'arboles, en un principio los costos o longitud del \'arbol.

		\item Cambie de posici\'on algunos nodos o terminales de la topolog\'ia, y observe c\'omo cambia la longitud del cladograma y los estados asignados a los nodos o la forma como los caracteres son mapeados, pruebe tambi\'en moviendo ramas completas.

	\end{enumerate}

	\item En \Pname{Mesquite} y \Pname{TNT} (para Windows) usted puede mover ramas completas; en \Pname{TNT} (para Windows) use el menu \Gui {Trees - view} y seleccione con el bot\'on izquierdo del rat\\'on el clado a mover (el clado queda marcado en rojo), se\~nale con el rat\'on el destino usando el bot\'on derecho.

	\item Desde \Pname{WinClada} o \Pname {TNT} (para Windows), haga una b\'usqueda de cladogramas para la matriz de vertebrados que contruy\'o en la pr\'actica de matrcies, use los par'ametros por omisi\'on (men\'u \Gui{Analize} / \Gui{heuristics}, \Gui{traditional search}).

	\item En \Pname{Mesquite} o \Pname{Figtree}:
	\begin{enumerate}
		\item Cambie el orden de los terminales sin cambiar la topolog\'ia (las relaciones entre terminales) del \'arbol.
		\item Coloque etiquetas en los nodos, y salve el \'arbol.
		\item Revise ese archivo en un editor de texto para determinar c\'omo se colocaron tanto las etiquetas como las longitudes.
		\item Abra el \'arbol con longitud de ramas.
	\end{enumerate}


	\item En \Pname{Mesquite} o \Pname{WinClada} abra el archivo \Datos{datos.conarbol.dat}:
	\begin{enumerate}

		\item Indique la cantidad de \'arboles presentes y la longitud de los mismos.

		\item Examine las agrupaciones obtenidas e identifique los caracteres que soportan los nodos.

		\item Mapee los caracteres en los \'arboles usando distintos tipos de optimizaciones: en \Pname{WinClada} ACCTRAN (=fast), DELTRAN (=slow) y no ambigua, En \Pname{TNT} (para Windows) (\Gui{Optimize $>$ Character $>$ Reconstructions} y seleccione un solo \'arbol y algunos caracteres) y en \Pname{Mesquite}: Dollo e Irreversible (C-S); revise c\'omo cambian las optimizaciones en los distintos nodos.

		\item Revise de nuevo el efecto en la longitud de los \'arboles de las siguentes modificaciones:
		\begin{enumerate}
			\item Cambie la ra\'iz del \'arbol.
			\item Cambie de posici\'on nodos y terminales.
			\item Seleccione un nodo y colapselo.
		\end{enumerate}
	
	\end{enumerate}

	

	\item \En \Pname{R}:
	\begin{enumerate}

		\item Lea el archivo con \'arboles \Datos{arbol.r.tre} use las instrucciones:
		\Cmd{library(ape); arbol <- read.tree(''arbol.r.tre'')}

		\item Grafique el objeto \Rdatos{arbol} use las instrucciones, que por defecto dibuja el \'arbol con longitud de ramas o como un filograma:
		\Cmd{plot(arbol)} o
		\Cmd{plot.phylo(arbol)}


		\item Para graficar el \'arbol sin incluir la longitud de ramas  use las instrucciones:
		\Cmd{plot(arbol,use.edge.length=FALSE)} 


		\item Para asignar las longitudes de las ramas como r\'otulos de los nodos y graficarlos use las instrucciones:
		\Cmd{arbol\$node.label<-arbol\$edge.length}
		\Cmd{plot(arbol,show.node.label=TRUE)}


		\item Para graficar el \'arbol con t\'itulo de gr\'afica, colores de ramas y tipo cladograma use las instrucciones:
		\Cmd{plot(arbol, type="cladogram",main="\'Arbol 1",
		edge.col=c("red","blue","cyan"))}


		\item Para obtener la longitud del \'arbol, lea la matriz de datos \Datos{matriz.r.phy} a un objeto que se llame \Rfunc{MatrizDatos} y use la funci\'on \Rfunc{parsimony()}:
		\Cmd{parsimony(arbol,MatrizDatos)}


	\end{enumerate}	


\end{enumerate}


\section*{Preguntas}

\subsection*{Pr\'actica}

\begin{enumerate}

	\item Al hacer una b\'usqueda por defecto ?`Usted y sus compan\~neros obtienen las mismas topolog\'ias y los mismos caracteres que soportan los grupos?.

	\item Al cambiar la aditividad/no aditividad de un caracter  ?`Cambia esto la forma como se mapean? Haga este ejercicio para varios caracteres, tanto binarios como multiestado.

	\item Si cambia la ra\'iz, ?`cambia la longitud? ?`Por qu\'e cree que se presenta el resultado que obtuvo? ?`Es (y por qu\'e) una ventaja o una desventaja?

	\item Examine sus caracteres y cambie su aditividad-no aditividad. ¿Cambia esto la forma como mapean? Haga este ejercicio para varios caracteres, binarios y multiestado.

	\item Si cambia la ra\'iz, ¿cambia la longitud? ¿Por qu\'e cree que se presenta el resultado que obtuvo? ¿Es (y por qu\'e) una ventaja o una desventaja?

	\item Dibuje el siguiente cladograma:\\ 
	\begin{small}
	\emph{(Lampreas (Tiburones (Esturi\'on Tele\'osteos) (Celacanto (Peces-pulmonados (Anfibios (Mam\'iferos (Tortugas (Lagartos (Cocodrilos Aves)))))))))}
	\end{small}

	\item Si la ra\'iz esta colocada entre Aves y Cocodrilos, ¿c\'omo es la topolog\'ia resultante?
	
	\item Convierta la topolog\'ia dibujada a notaci\'on de par\'entesis.
\end{enumerate}

\subsection*{Generales}

Aparte de no tener un an'alisis expl\'icito, existe una gran diferencia entre los \'arboles filogen'eticos actuales y sus ''equivalentes'' usados por algunos tax\'onomos (por ejemplo, Haeckel, Romer, etc.) ?`Es usted capaz de descubrirla? \textbf{Clave}: intente dibujar alguno de esos \'arboles al estilo actual.

\section*{Literatura recomendada}

Page \& Holmes, 1998 [El cap\'itulo 2 esta dedicado a los \'arboles y presenta muy buenas ilustraciones, Tambi\'en puede consultar la p\'agina de docencia de Rod Page (\url{taxonomy.zoology.gla.ac.uk/teaching/index.html})].

%% todo: revisar validez de pagina rod page docencia

\chapter{B'usquedas mediante parsimonia}
\section*{Introducci'on}
Una vez se ha construido la matriz de caracteres; el  problema es la selecci'on de los cladogramas. Aunque Hennig (1968) presenta argumentos para favorecer la agrupaci'on por sinapomorf'ias, no fue muy expl'icito en c'omo obtener los cladogramas. Camin \& Sokal (1965) fueron quiz'as los primeros en sugerir el uso de parsimonia para hacer tal selecci'on; as\'i el cladograma preferido es aquel que minimiza la cantidad de transformaciones (el m\'as parsimonioso). Posteriormente, el m'etodo fue generalizado usando la optimizaci'on de Wagner y la de Fitch (Wagner, 1961; Kluge \& Farris, 1969; Farris, 1970; Farris et al., 1970; Fitch, 1971).\\
La b'usqueda de cladogramas se complica con la adici'on de terminales (t'ecnicamente es un problema NP-completo), y por ello, soluciones exactas s'olo son posibles con pocos terminales (aprox. 20). Sin embargo, se han ideado gran cantidad de heur'isticas, que aunque no proporcionan con certeza la soluci'on 'optima, proveen resultados que son dif'iciles de superar, al menos con las tecnolog'ias actuales.\\
La forma m'as sencilla de elaborar cladogramas es usando el algoritmo de Wagner:\index{B\'usqueda!algoritmo de Wagner} como el orden de entrada de los taxa afecta la topolog'ia, se realiza una aleatorizaci'on de tal secuencia de entrada (Dayoff, 1969),\index{B\'usqueda!RAS, random addition sequence} la cual puede estar seguida de una permutaci'on final de las ramas; sin embargo, este 'ultimo paso para matrices \textbf{muy} grandes consume gran cantidad de tiempo. Esto se debe repetir m'ultiples veces para evitar caer en ''\'optimos locales''. Con este m'etodo es posible una soluci'on 'optima incluso para matrices de 80 a 100 terminales. Problemas m'as grandes requieren nuevas estrategias, algunas de ellas derivadas de la cristalizaci'on simulada, aceptando moment'aneamente cladogramas sub'optimos para iniciar desde ellos la permutaci'on de ramas. Otros utilizan combinaciones bien sea entre b'usquedas exhaustivas o entre b'usquedas sobre reducciones de la matriz.\\
Una nueva ventana para las matrices cada vez m\'as grandes (por ejemplo matrices moleculares con m'as de 1000 terminales), es tratar de identificar el acuerdo entre distintas r'eplicas de b'usquedas parciales, en vez de buscar la soluci'on 'optima (e.g. Farris et al., 1996; Farris, 1997; Goloboff, 1997b; Goloboff \& Farris, 2001). Dado que la cantidad de estudios con un gran n'umero de terminales es a'un muy peque~na, estos m'etodos hasta ahora simplemente se han presentado y no hay mucha discusi'on alrededor de ellos.\index{B\'usqueda!nuevas metodolog\'ias}
\section{T'ecnicas}
\index{B\'usqueda!t'ecnicas}
El algoritmo de Wagner es la base para las b'usquedas actuales. Para evitar el problema del orden de entrada de los datos, estos se unen al azar. La mayor parte de los programas actuales tiene esta opci'on: inician con una \textbf{semilla} determinada para el generado de n'umero aleatorio y aseguran que la b'usqueda sea exactamente igual a otra que tenga la misma semilla. Una vez construido un cladograma, este suele ser sometido a permutaci'on de ramas para mejorar su calidad. B'asicamente se toma un nodo (sub'arbol) y es eliminado del cladograma principal, luego se prueba si al unirlo en diferentes lugares del cladograma principal disminuye la longitud con respecto al cladograma original. Se puede permutar ramas de varias formas; las m'as comunes son unir el nodo a las diversas ramas del cladograma principal (subpoda y replantado, SPR por sus siglas en ingl'es)\index{B\'usqueda!SPR}, o intentar otros puntos de uni'on dentro del sub'arbol, sin cambiar su topolog'ia (bisecci'on y reconexi'on de arboles, TBR). En general, la mayor parte de los programas utilizan TBR,\index{B\'usqueda!TBR} puesto que el tiempo de permutaci'on entre ambas t'ecnicas es casi igual y TBR es mucho m'as eficiente.\\
Durante la permutaci'on pueden retenerse todos los 'arboles encontrados debidos a empates (''estrategia PAUP*'', Rice et al., 1997), o s'olo uno o pocos 'arboles, ignorando los empates (''estrategia NONA'', Goloboff, 1999; Nixon, 1999; Quickle et al., 2001). La estrategia NONA emp'iricamente provee los mejores resultados, al maximizar el n'umero de posibles puntos de arranque y evitar el problema de detenerse en \emph{islas} de m'ultiples (en ocasiones miles) de 'arboles. La estrategia NONA suele acompa'narse de una permutaci'on final de ramas, m'as extensa, de los mejores 'arboles encontrados.\\
Para menos de 80-100 taxa es suficiente con  realizar m'ultiples (100 suele bastar en la mayor'ia de los casos) b'usquedas de 'arboles de Wagner con permutado de ramas y una permutaci'on final de ramas. Aunque es posible que as'i se ignoren muchos 'arboles igualmente 'optimos, pero el resultado final, expresado en un consenso estricto, es dif'icil que cambie (Farris et al., 1996; Farris, 1997, Goloboff \& Farris, 2001).\\
En problemas m'as complejos se requiere utilizar t'ecnicas m'as sofisticadas para obtener resultados satisfactorios. La m'as sencilla es el rastrillo o pi~n'on (\textit{ratchet}\index{B\'usqueda!ratchet} en ingl'es) de Nixon (Nixon, 1999; tambi'en Quickle, 2001), la cual es una forma simple de implementar una cristalizaci'on simulada. El m'etodo consiste en usar un 'arbol ya elaborado (por ejemplo con Wagner+TBR), perturbar la matriz de datos (con eliminaci'on de caracteres o repesado), hacer permutaci'on de ramas del 'arbol para obtener el 'optimo de la nueva matriz, volver la matriz a su estado original y buscar el 'arbol 'optimo con permutado de ramas (todo ese proceso es una iteraci'on, la cual se repite \textbf{n} veces). El rastrillo es eficiente usando solo unos pocos 'arboles por iteraci'on y permutando una cantidad intermedia de caracteres (entre 10-25\%), y mejora dr'asticamente el ajuste de los cladogramas en las primeras iteraciones (v'ease Nixon, 1999).\\
Para producir nuevas mejoras en el ajuste de cladogramas, los m'etodos m'as eficientes parecen ser la ''deriva de 'arboles'', que es una implementaci'on m'as expl'icita de la cristalizaci'on simulada (es decir aceptar soluciones ligeramente sub'optimas con una determinada probabilidad, y a medida que el an'alisis avanza, se disminuye la probabilidad de aceptaci'on de los sub'optimos), y la fusi'on de 'arboles, que utiliza lo mejor de diferentes soluciones. Una revisi'on completa de estos m'etodos est'a en Goloboff (1999).
\section{Materiales}
\noindent
Matriz chica o ''normal'' (datos.chica.dat).\\
Matriz grande (datos.ratchet.dat).\\
Instrucciones para CREAR un archivo para ratchet en \Pname{PAUP*} usando \Pname{TNT} (pauprat.run).
\section{M'etodos}
\noindent
En todos los casos, salve los cladogramas, reporte el n'umero y longitud de los cladogramas y el tiempo en que se realiz'o la b'usqueda.\\
\textbf{En \Pname{WinClada}, \Pname{TNT} y \Pname{PAUP*}:}\\
(1) Ejecute la matriz datos.chica.dat y haga una b'usqueda por omisi'on (tal como viene definida en el programa).\\
(2) Repita la b'usqueda, pero ahora con 5 r'eplicas, 100 'arboles por r'eplica. Haga, una segunda b'usqueda con 10 r'eplicas, 1 'arbol por r'eplica y por 'ultimo 100 r'eplicas, 1 'arbol por r'eplica.\\
\textbf{En \Pname{WinClada} y \Pname{TNT}:}\\
(3) Abra la matriz datos.ratchet.dat, haga una b'usqueda con estrategia NONA (100 r\'eplicas, 1 'arbol por r'eplica).\\
(4) Haga dos b'usquedas secuenciales (2 r'eplicas) de ratchet con 50 iteraciones.\\
(5) Haga 20 b'usquedas secuenciales de ratchet con 5 iteraciones cada una.\\
(6) Ejecute el archivo de macros pauprat.run usando los par'ametros 10 5 (10 r'eplicas y 5\% de corte).
\\
\textbf{En \Pname{PAUP*}:}\\
(7) Haga una b'usqueda tipo ratchet, usando el archivo de instrucciones pauprat generado por TNT.\\
(8) Cambie los valores de corte en el paso (6) a un porcentaje mayor y repita el paso (7).
\subsection{Programas}
\noindent
La mejor opci'on para programas gratuitos es \Pname{NONA}; este es un programa completo de an'alisis clad'istico desarrollado para Windows/DOS y con copias disponibles para MacOSX y Linux,  es bastante r'apido, adem'as de tener implementado ratchet. Adicionalmente, el programa puede manejarse como buscador con \Pname{WinClada} (para Windows), es buena idea que el lector(a) se familiarice con \Pname{NONA}: las b'usquedas son m'as eficientes desde la l'inea de comandos.\\
\Pname{PAUP*} y \Pname{TNT} est'an disponibles como ejecutables en varias plataformas (Windows, Mac y Linux). \Pname{PAUP*} no solo usa parsimonia sino distancias y m'axima verosimilitud, aunque para parsimonia es menos vers'atil que NONA. TNT est'a dise~nado para b'usquedas exhaustivas en matrices grandes y su velocidad y sistema de macros son sorprendentes; pero, por lo menos hasta el momento, no hace b'usquedas mediante ML.
\subsection{Comandos}
\noindent
Revise el ap\'endice~\ref{ch:programas} (programas de c'omputo) para ver de forma m'as detallada los distintos comandos utilizados (p\'agina \pageref{ch:programas}). \Pname{NONA}, \Pname{TNT} y \Pname{PAUP*} cuentan con ayuda en l'inea (Comando \Cmd{help}). Para las b'usquedas con \Pname{WinClada}, recurra al men'u \Gui{Analize}, en las entradas \Gui{heuristics} y \Gui{ratchet}. Para b'usquedas con \Pname{NONA}, el comando m'as usado es \Cmd{mult*}  para las b'usquedas iniciales, \Cmd{max*} para permutar ramas (requiere 'arboles) y \Cmd{nix*}  para ratchet. En \Pname{TNT} tambi'en se puede usar \Cmd{mult}; la permutaci'on de ramas es con \Cmd{bbreak}. \Pname{PAUP*} debe estar en b'usquedas con parsimonia \Cmd{set criterion=parsimony}, y la b'usqueda se usa haciendo \Cmd{hsearch}  tanto para 'arboles de Wagner como para permutar ramas; en este 'ultimo caso use \Cmd{hsearch start=current}.\\
Para los archivos de macros use la instrucci'on \Cmd{run} seguida del nombre del archivo; en este caso pauprat.run y los par'ametros (\Cmd{run pauprat.run 10 5}); \Pname{TNT} usa pesos de 1 y 2 en el archivo de salida, pauprat.

\section{Preguntas}
\subsection{Pr'actica}
\noindent
De los diferentes programas usados, ?`C'ual estima usted que es el 'optimo? Explique las razones de su selecci'on.\\
?`C'ual cree usted que ser'ia(n) el(los) criterio(s) para seleccionar entre los diferentes programas?\\ 
Elabore una tabla usando sus resultados y los de sus compa'neros. Para cada matriz, ?`en qu'e clase de b'usqueda se obtuvo el mejor resultado?, ?`c'ual fue el tiempo en que se obtuvo dicho resultado?
\subsection{Generales}
\noindent
Dado que con una t\'ecnica heur\'istica existe el riesgo de no obtener el \'arbol m\'as corto ?`como justificar\'ia usted la b\'usqueda realizada?\\
En este laboratorio solo se utilizaron algunos tipos de b'usquedas posibles y algunos de los posibles comandos para cada programa. Trate de encontrar otros comandos de b'usqueda en estos programas u otros par'ametros para los comandos usados en la pr'actica.
\section{Literatura recomendada}
\noindent
deLaet, 2005 [Para quienes les gusta programar. Una introducci'on a los algoritmos usados en b'usqueda de 'arboles].\\
Goloboff, 1999 [En este art'iculo se proponen algunos de los m'etodos implementados en \Pname{TNT}].\\
Goloboff \& Farris, 2001 [Una implementaci'on y puntos de vista acerca de las b'usquedas r'apidas].\\
Nixon, 1999 [Propuesta de ratchet y comparaci'on de la estrategia PAUP contra la estrategia NONA].\\
Swofford et al., 1996 [Un cap'itulo general sobre an'alisis filogen'etico. Con una introducci'on a la b'usqueda y optimizaci'on de 'arboles].
\input{pesaje.tex}
\chapter{Consensos}
\section*{Introducci'on}
Seguramente, por sus pr'acticas anteriores y por la literatura consultada, usted ha notado que en muchas ocasiones se produce m'as de un cladograma como respuesta. Para resumir la informaci'on contenida en los diferentes cladogramas se puede usar un consenso, el cual se puede obtener mediante varios procedimientos. Los 'arboles de consenso contienen informaci'on sobre los agrupamientos en los diferentes 'arboles obtenidos. Swofford (1991) y Nixon \& Carpenter (1996) ofrecen una discusi'on extensa sobre los 'arboles de consenso con dos visiones diferentes. Es importante recalcar que la topolog'ia del consenso es un resumen de los cladogramas y \textbf{no} es una \textbf{hip'otesis de filogenia}.\\
En el consenso estricto, solamente los nodos compartidos por todos los cladogramas son incluidos en el resumen final. Este m'etodo es el m'as conservativo. Otros dos tipos de consensos, de Bremer y de Nelson, son muy similares al consenso estricto, s'olo difieren de este en casos muy particulares y en general no son usados.\\
El consenso de Adams se basa en operaciones de conjuntos entre los nodos; es muy 'util para mostrar cu'ales son los taxa que producen inestabilidad en el cladograma, pero puede producir nodos que no se encuentran en ninguno de los cladogramas originales. Kearny (2002) ofrece una buena discusi'on sobre c'omo combinar los resultados del consenso estricto y el de Adams.\\
Los consensos de la mayor'ia son muy populares, especialmente en los an'alisis moleculares, aunque su uso es \textbf{muy} discutible (Sharkey \& Leathers, 2001). En este tipo de consenso se hace un conteo de las veces que el nodo aparece en los diferentes 'arboles: si el nodo aparece en al menos la mitad de los 'arboles, este es incluido en el consenso. Es posible hacer cortes m'as estrictos. Es una convenci'on colocar en forma de porcentaje la cantidad de veces en las que el nodo apareci'o.\\
Como se mencion'o en la introducci'on de la pr'actica de b'usquedas, los m'etodos de consenso tambi'en se est'an usando en la actualidad para respuestas parciales, como b'usquedas de \textit{jackknife}, o el doble consenso de Goloboff \& Farris (2001); o como resultados en el caso de los an'alisis bayesianos que usan el 'arbol obtenido en el consenso de la mayor'ia (Huelsenbeck et al. 2001, 2002).
\section{T'ecnicas}
En la mayor parte de los programas los consensos estrictos y de la mayor'ia est'an implementados (por ejemplo, \Pname{TNT}, \Pname{PAUP*}, \Pname{WinClada} o \Pname{Component}). En algunos casos, la implementaci'on del consenso de la mayor'ia es s'olo hasta el 50\%, y el usuario decide que tan estricto hace el corte, eliminando los nodos que est'en por debajo del valor de corte (un consenso estricto s'olo retendr'ia los nodos con soporte del 100\%).\\
En otros casos, los programas pueden elaborar el consenso de la mayor'ia, pero no reportan los porcentajes de aparici'on en los nodos (por ejemplo en \Pname{TNT}), o simplemente los visualizan pero no salvan el 'arbol con los porcentajes incluidos (\Pname{TNT}). Es importante que si va a usar esta clase de consenso, eval'ue c'omo recuperar los reportes de frecuencias si usa \Pname{TNT}.\\
Finalmente hay que alertar un poco acerca de la resoluci'on de los cladogramas usados para generar los consensos. La mayor parte de los programas usa los 'arboles perfectamente dicot'omicos, por lo que pueden incluirse ramas no soportadas; as'i, al hacer un consenso es necesario eliminar tales 'arboles con ramas no soportadas. La mayor'ia de los programas actuales incluyen opciones para controlar la salida: incluir o no incluir 'arboles con nodos de longitud cero (por ejemplo \Pname{TNT}, \Pname{NONA}, \Pname{WinClada}, \Pname{MacClade} y \Pname{PAUP*}), otros no (\Pname{Component}).
\section{Materiales}
\noindent
Matriz de datos (datos.consenso.dat).
\section{M'etodos}
\noindent
\textbf{En \Pname{WinClada}, \Pname{PAUP*} y \Pname{TNT}:} \\
(1) Abra la matriz de datos, y realice una b'usqueda convencional. Guarde los 'arboles encontrados.\\
(2) Realice y salve un consenso estricto, un consenso de la mayor'ia al 50\%, 75\% y 90\% de corte. En todos los casos reporte el n'umero de nodos presentes en el 'arbol, y compare los grupos encontrados. Recuerde anotar las frecuencias de los grupos pues \Pname{TNT} \textbf{no} las salva.\\
\textbf{En \Pname{PAUP*}}:\\
(3) Siga el mismo esquema que en (1) y (2), realice una b'usqueda y calcule primero el consenso por defecto y posteriormente los consensos estricto y de Adams; gu'ardelos en un archivo.\\
\textbf{En \Pname{Component}:}\\
(4) Abra los 'arboles encontrados en (1) o en (3). Calcule y salve el consenso estricto, de la mayor'ia y de Adams. Recuerde que \Pname{Component} no verifica si los nodos tienen longitud cero.
\subsection{Programas}
\noindent
\Pname{WinClada}, \Pname{PAUP*}, \Pname{TNT}, \Pname{Component}.
\subsection{Comandos}
Para las instrucciones de b'usqueda revise la pr'actica correspondiente.\\
Para \Pname{WinClada}  use los men'us correspondientes en la secci'on de \Gui{winclados}; el men'u \Gui{Trees} tiene la entrada \Gui{consensus compromise}, donde hay la posibilidad de hacer consenso estricto, consenso estricto eliminando nodos no soportados (nelsen, que no debe confundirse con el consenso de Nelson) y consenso de la mayor'ia\footnote{una vez calculado el consenso de la mayor'ia, \Pname{WinClada} tiene algunos problemas posteriores en el manejo de los 'arboles en la forma como los grafica. Por ejemplo, los \textit{Hashmarks} no se pueden activar (no son dibujados), y modificar la topolog'ia produce cambios inesperados en las frecuencias de los nodos.}. Para salvar los 'arboles siga las instrucciones previas usadas en la pr'actica de b'usquedas.\\
En \Pname{TNT} use la instrucci'on \Cmd{nelsen} para obtener el consenso estricto, si desea que el consenso sea el 'ultimo 'arbol, use \Cmd{nelsen*}; para el consenso de la mayor'ia, use la instrucci'on \Cmd{majority} y \Cmd{majority*}, respectivamente. Tambi'en puede usar \Cmd{save \{strict\}}  o   \Cmd{save \{majority\}} para guardar los 'arboles de consenso directamente, despu'es de calculados. En todos los casos \textbf{debe} haber abierto previamente el archivo de 'arboles.\\
\Pname{PAUP*}: use la instrucci'on \Cmd{contree}. Use en caso de duda el nombre del comando y el signo \textbf{?} [\Cmd{contree ?}] para que eval'ue las opciones del comando; si desea guardar el consenso directamente en un archivo de 'arboles \textbf{debe} hacerlo desde esta instrucci'on.
\section{Preguntas}
\subsection{Pr'actica}
\noindent
Compare las topolog'ias de los 'arboles encontrados: ?`difieren los resultados entre los programas? Revise si las frecuencias de los nodos comunes son iguales en los diferentes resultados.\\
Al asignar las transformaciones (sinapomorf'ias) de cada nodo sobre el consenso, ?`c'omo lo har'ia usted? Compare su aseveraci'on con la implementada en los programas (recuerde la pr'actica de matrices).
\subsection{Generales}
\noindent
?`Recomendar'ia usted el uso de consensos de mayor'ia como herramienta para resumir la informaci'on de los 'arboles iniciales?
\section{Literatura recomendada}
\noindent
Goloboff \& Farris, 2001 [Presenta las t'ecnicas de consensos r'apidos].\\
Miyamoto, 1985 [presenta una cr'itica hacia la interpretaci'on de los consensos en clasificaciones].\\
Nixon \& Carpenter, 1996 [Una discusi'on cl'asica sobre consensos].\\
Sharkey \& Leathers, 2001 [Una excelente cr'itica al uso y abuso del consenso de mayor'ia].
\input{soporte.pars.tex}
\chapter{Alineamiento de ADN}
\section*{Introducci'on}
\label{ch:alinear}
Aunque los datos moleculares pueden proporcionar grandes cantidades de caracteres, presentan retos 
en la asignaci'on de homolog'ias al tratarse exactamente de las mismas bases en todas las secuencias; adem'as 
en muchas ocasiones las secuencias tienen un tama~no desigual. \index{Alineamiento!definici\'on}
Para resolver estos problemas se han ideado los m'etodos de alineamiento, que buscan recuperar las posibles homolog'ias 
dentro de diferentes secuencias, al utilizar una matriz de transformaciones, entre las cuatro bases y los gaps o 
InDels (inserci'on/eliminaci'on); as\'i, colocando gaps las dos secuencias son equivalentes en longitud y  posiciones
 para las bases; por ejemplo, compare la secuencia 1x con cualquiera de los alineamientos 2.
 \footnote{Ejemplo modificado de Siddall, \href{}http://research.amnh.org/~siddall/methods.}\\
\\
\begin{small}
1x: \Cmd{a c t T C C g A A T T T g g c t a c T T C C g A A t T T g G c}\\
2x: \Cmd{a c t C G A T T G C C T A C T C G A T T G C C}\\
2a: \Cmd{a c t - - C g A - - T T g c c t a c T - C - g A - t T - g C c}\\
2b: \Cmd{a c t - C - g A - T - T g c c t a c - T - C g A - t - T g C c}\\
2c: \Cmd{a c t - C - g - A T T - g c c t a c - T - C g - A t - T g C c}\\
\end{small}
\\
Ya que se necesitan m'as de tres secuencias para un an'alisis filogen'etico, es necesario el alineamiento simult'aneo de 
m'ultiples secuencias. El primer m'etodo usado es el alineamiento m'ultiple que usa un 'arbol \textbf{gu'ia} con las 
secuencias en los terminales; a partir de este 'arbol se hace el primer alineamiento para el par m'as 
cercano\footnote{En realidad es el par m'as similar, ya que se usa una t'ecnica de distancia para obtener el 'arbol gu'ia.}, 
luego optimiza los gaps, posteriormente se alinea con la siguiente terminal que sea la m'as cercana y el proceso se repite 
hasta llegar a la base; la idea es parecida a la optimizaci'on en un an'alisis filogen'etico.\\
Una nueva idea, usada principalmente por los cladistas, es el uso de optimizaci'on directa (Wheeler, 1996), donde el 
objetivo no es construir el alineamiento, sino directamente el cladograma; en este caso, se hacen los alineamientos locales 
(como en el alineamiento m'ultiple) y los gaps son considerados transformaciones, es decir se colocan temporalmente en 
los nodos. Wheeler (1996) argumenta que implementa una idea de homolog'ia din'amica acorde con el an'alisis filogen'etico. 
En la optimizaci'on de estados fijos (Wheeler, 1999) se usa la secuencia completa, y la matriz de transformaciones s'olo 
sirve para asignar costos, pero no estados en el nodo. \index{Alineamiento!m\'ultiple} Aunque es intuitivamente muy 
interesante, no se ha usado en la pr'actica. Cualquiera que sea la idea para realizar los an'alisis moleculares, los 
resultados dependen de la matriz de transformaci'on inicial; Wheeler (1995) desarroll\'o una metodolog'ia para comparar 
los par'ametros de las transformaciones, conocida como ''an'alisis de sensibilidad''.
\index{Alineamiento!an'alisis de sensibilidad}
Bajo esta idea, se hacen diferentes an'alisis con diferentes matrices de transformaci'on, y se selecciona aquella que 
maximice alg'un criterio previamente seleccionado (ya sean las topolog'ias, como el costo de las transformaciones); hasta 
ahora, esta idea solo se ha utilizado en el contexto de an'alisis con m'ultiples genes o entre genes y morfolog'ia 
simult'aneamente.
\section{T'ecnicas}
Algunos programas (como \Pname{Clustal}) usan un solo 'arbol gu'ia derivado del c'alculo de la distancia entre las 
secuencias; como todos los m'etodos basados en distancias, es dependiente del orden de entrada de los datos; otros 
programas (como \Pname{MALIGN} o \Pname{POY}) construyen distintos 'arboles y prefieren el alineamiento que tenga el menor 
costo. Aunque sean m'as lentos, \Pname{POY} y \Pname{MALIGN} producen mejores resultados (en cuanto a la calidad del 
alineamiento con miras a evaluar la filogenia) que \Pname{Clustal}. Giribet et al. (2002) recomiendan que si se usa 
\Pname{Clustal}, se prueben diferentes secuencias de entrada de los datos. Es importante recalcar que el 'arbol usado por 
\Pname{Clustal} y por \Pname{MALIGN} es s'olo un 'arbol para optimizar las secuencias, no un 'arbol filogen'etico como tal; 
mientras que el 'arbol generado por \Pname{POY} es un 'arbol de alineamiento y a la vez es un 'arbol filogen'etico. Giribet 
et al. (2002) enfatizan que de usar los alineamientos, los costos usados para el alineamiento deben ser los mismos usados 
para el an'alisis filogen'etico. Al asignar los costos, hay que tener en cuenta la desigualdad triangular. Es decir, que el 
costo de una transformaci'on nunca puede ser mayor al de una transformaci'on equivalente, pero que tome otros estados. 
Un ejemplo clarifica esto: si las transiciones tienen un costo de 3 y las transversiones de 1, la transformaci'on 
A$\rightarrow$ G, tendr'ia un costo de 3, pero si se hace de la forma A$\rightarrow$ T$\rightarrow$ G, el costo ser'ia de 
2 (dos transversiones). Esto har'ia que a los nodos se les pudiesen asignar estados no observados en los terminales.\\
Para facilitar tanto la velocidad de alineamiento como las asignaciones de homolog'ia, al usar \Pname{POY} se pueden 
(y preferencialmente se deben) dividir las secuencias analizadas en peque~nas secuencias o marcos; definidos mediante 
iniciadores universales, estructura secundaria o motivos conservados. Cuando las secuencias son muy desiguales en estos 
segmentos, muchas veces se prefiere eliminarlas a analizarlas (o se hace un an'alisis exploratorio que incluya esas 
secuencias).
\section{Materiales}
\noindent
Secuencias para:\\
\Pname{MALIGN} (datos.malign.dat).\\
\Pname{FASTA} (datos.fas.dat).\\
Par'ametros para \Pname{MALIGN} (param.txt).\\
Archivo de instrucciones para \Pname{POY} (lotes.poy).\\
Matriz morfol'ogica (datos.ss.dat).
\section{M'etodos}
\noindent
\textbf{En \Pname{MALIGN}:}\\
(1) Abra en un editor los archivos datos.malign.dat y param.txt.\\
(2) Eval'ue qu'e clase de b'usqueda se producir'a con esos par'ametros.\\
(3) Ejecute \Pname{MALIGN} con los par'ametros predefinidos. ?`Cu'al es la relaci'on transici'on/ transversi'on definida?\\
(4) Modifique el archivo de par'ametros de tal forma que: 
 (a) Tenga una relaci'on ts/tv de 1/2,
 (b) Tenga una relaci'on ts/tv de 1/5.\\
(5) Ejecute nuevamente \Pname{MALIGN} con estas modificaciones en los par'ametros.\\
(6) Cambie los par'ametros dando a los gaps un costo del doble del costo de las transversiones.\\
(7) Modifique el archivo de par'ametros y esta vez produzca la salida para buscar el 'arbol en \Pname{PAUP}.\\
(8) Revise la salida de cada una de las modificaciones y eval'ue el tama~no de la matriz resultante en cada caso\\
(9) Para cada alineamiento, busque el 'arbol m'as parsimonioso con el programa de su elecci'on.\\
 \textbf{En \Pname{POY}:}\\
(10) Abra en un editor el archivo de procesos lotes.poy ?`Qu'e b'usqueda esta implementada en \Pname{POY}?\\
(11) Ejecute el programa y llame al archivo de instrucciones (tambi\'en puede usar el modo interactivo y ejecutar instrucci\'on por  instrucci\'on).\\
(12) Modifique el archivo para usar la matriz de datos morfol'ogicos y realizar un an'alisis de evidencia total.\\
(13) Modifique el archivo para hacer un alineamiento tipo FSO y pese la matriz morfol'ogica 2 veces el valor de los datos
 moleculares.\\
(14) Modifique el archivo para hacer un alineamiento tipo OA.\\
(15) Cambie los par'ametros de tal manera que los gaps valgan el doble que las sustituciones.\\
(16) Genere los respectivos alineamientos y busque el 'arbol m'as parsimonioso.\\
(17) Modifique el archivo de par'ametros y esta vez produzca la salida para hen86/winclada.\\
(18) Compare el 'arbol resultado de la alineaci'on con \Pname{POY} y con \Pname{MALIGN}.
\subsection{Programas}
Si desea hacer un alineamiento m'ultiple r'apido use \Pname{Clustal}, pero si el objetivo es la filogenia sugerida por las 
secuencias es mejor usar \Pname{MALIGN} o \Pname{POY}, ya que estos tienen en cuenta el 'arbol final. Puede hacer una 
exploraci'on en \Pname{Clustal} y posteriormente llevar sus marcos como archivos separados para ser analizados con \Pname{POY}.
\subsection{Comandos}
\Pname{MALIGN} utiliza un archivo de par'ametros para configurar el alineamiento; algunos de los par'ametros m'as 
importantes son \Cmd{changecost} para asignar el costo de una transformaci'on y \Cmd{gap} para asignar el costo de 
adicionar un gap. \Cmd{matrix} asigna una matriz de costos del alineamiento. Las b'usquedas pueden ser \Cmd{quick}, 
una b'usqueda r'apida, o \Cmd{build}, junto con \Cmd{aspr} o junto con \Cmd{atbr}, que permutan ramas, 
mientras que \Cmd{keepalignment} indica el n\'umero de alineamientos igualmente 'optimos que se van a retener. Cuanto m'as grande 
m'as lento ser'a su an'alisis. Con \Cmd{hennig86}, o \Cmd{nexus}, da el formato de salida para \Pname{NONA} o 
para \Pname{PAUP*}. En \Pname{POY} los par'ametros son colocados despu\'es de la instrucci\'on y en par\'entesis; los archivo de entrada (datos o intrucciones) se llaman desde la l\'inea de comandos \Cmd{poy intrucciones} o con \Cmd{read("morph.txt")} de manera interactiva; con
(**aqui faltan mas comandos**)
 
\Cmd{-change x} se asigna el costo de las transformaciones; con \Cmd{-gap x} el costo de los gaps; tambi'en puede usar 
una matriz expl\'icita de transformaci'on entre estados; a diferencia de la usada en \Pname{MALIGN}, la matriz de 
\Pname{POY} usa cinco estados, donde el quinto \textit{estado} es gap; 
\Cmd{build (100)} hace una b\'usqueda r\'apida 
\Cmd{} hace una b'usqueda heur'istica 
con tbr. \Cmd{} o \Cmd{} indican cuantos 'arboles se guardan en la b'usqueda.\\ 


Una de las grandes ventajas 
de \Pname{POY} es que puede ser usado en un \textit{cluster} (conjunto de computadoras enlazadas para realizar el mismo proceso).
\section{Preguntas}
\subsection{Pr'actica}
\noindent
Compare los 'arboles de \Pname{MALIGN}, \Pname{POY} y los obtenidos en un an'alisis clad'istico con el programa de su  preferencia. ?`Son similares los resultados?\\
Compare sus resultados con los de sus compa~neros. ?`C'omo son las longitudes de los 'arboles (las reportadas por \Pname{POY}) y las topolog'ias?\\
Dados los diferentes costos usados en el an'alisis simult'aneo de morfolog'ia y datos moleculares, 
 ?`cu'al cree usted que es el resultado 'optimo y por qu'e?
\subsection{Generales}
\noindent
Existe una disputa sobre una relaci'on entre los juegos de costos y el soporte. Dados sus conocimientos, escriba un 
peque~no ensayo donde indique sus ideas, su posici'on y sus argumentos en esta discusi'on.
\section{Literatura recomendada}
\noindent
Frost et al., 2001 [Un art'iculo emp'irico sobre el an'alisis de sensibilidad].\\
Giribert, 2003 [Revisa la exploraci'on de los resultados del an'alisis de sensibilidad vs. soporte de clados].\\
Giribet \& Wheeler, 1999 [Uno de los pocos art'iculos que discute expl'icitamente el tema de los gaps].\\
Wheeler, 1995 [La propuesta del an'alisis de sensibilidad para la asignaci'on de par'ametros en los alineamientos].
Wheeler etal., 2006 [Una gu\'ia completa para POY].
\chapter{Modelos de evoluci'on}
\section*{Introducci'on}
\label{ch:molecular}
Los sistem'aticos moleculares se enfrentan ante un conjunto de datos distinto en algunos aspectos del que enfrentan los morf'ologos:  un mismo tipo de estados (ACGT) se encuentra repetido a lo largo de toda la matriz de datos. En general, las ideas desarrolladas para analizar los datos moleculares tienen una aproximaci'on estad'istica, dado el origen de muchos de los an'alisis moleculares en biolog'ia molecular o en gen'etica de poblaciones. Bajo la estimaci'on estad'istica se asume un modelo que genera las secuencias de ADN y con base en el modelo se estima qu'e tanto se ajustan las hip'otesis filogen'eticas a los datos. A ese procedimiento se la conoce como \textbf{m'axima verosimilitud} (\textit{maximum likelihood} en la literatura en ingl'es).\\
Los modelos moleculares se basan en la asignaci'on de la probabilidad de transformar una base cualquiera en otra base, incluida esa misma base. El c'alculo de esas probabilidades est'a influenciado por diferentes par'ametros, que, en principio, deber'ian ser estimados de los datos, pero tambi'en se han usado par'ametros predefinidos, que quiz'as no la mejor opci'on.\\
El modelo con le menor n\'umero de par\'ametros es conocido como JC 'o JC69, por los autores y el a~no en que fue propuesto (Jukes y Cantor, 1969); se asume que la probabilidad de una base para transformarse en otra es siempre igual y que la frecuencia de las bases es igual (al menos al inicio de la evoluci'on de los organismos). De ah'i en adelante pueden agregarse m'as par'ametros que hacen m'as complejo el modelo en diferentes direcciones. Puede asumirse diferencia entre cada tipo de base (ya sea qu'imicamente AG-CT, o cada una por separado), diferencia de las probabilidades basada en la proporci'on de cada base, tener en cuenta o no los sitios invariantes, asumir que algunos sitios son m'as propensos a cambiar que otros (funci'on $\Gamma$) y, finalmente, si existe o no un reloj molecular.\\
Swofford et al. (1996) es una referencia b'asica para comprender el desarrollo actual de los modelos. Page \& Holmes (1998) ofrecen un cap'itulo ilustrativo sobre el tema. La idea general de modelos tambi'en se aplica para la evoluci'on prote'ica, pero dado que casi siempre se tiene tanto la secuencia de amino'acidos como la nucleot'idica, esta 'ultima es la preferida al explotar m'as directamente la informaci'on (al menos a nivel de variaci'on).\\
Es de notar que las probabilidades de los datos (la verosimilitud) suelen ser muy peque~nas, por lo que para facilitar los c'alculos se utiliza el negativo del logaritmo natural de la verosimilitud, que es el valor reportado por los programas as'i como en la literatura.
\section{T'ecnicas}
Debido al aumento en el n'umero de los par'ametros,los modelos puedan explicar m'as satisfactoriamente los datos (pero esto no implica que los modelos sean \emph{\textbf{m'as} realistas}); por lo que la selecci'on de un modelo no puede basarse en que el modelo mejora la probabilidad de los datos, sino si la mejora es (o no es) estad'isticamente significativa, dado el n'umero de par'ametros usados en el modelo.\\
Existen dos aproximaciones b'asicas para este problema. Dado que la mayor parte de los modelos son una especializaci'on de otros modelos m'as sencillos, es decir son modelos anidados, en los cuales el modelo m'as sencillo es s'olo un caso especial del modelo m'as complejo, se puede hacer una prueba jer'arquica de verosimilitud (hLRT\footnote
{\begin{equation}
\delta = -2 log \frac{Max[L0(modelo-nulo|datos]}{Max[L1(modelo-alternativo|datos]} 
\end{equation}
}) que compara la proporci'on en la que se increment'a la verosimilitud al agregar el par'ametro; se asume una distribuci'on $\chi ^2$  para la distribuci'on de esa proporci'on, tomando el n'umero de par'ametros extra como los grados de libertad. El problema de esta prueba es que no es claro si la distribuci'on $\chi ^2$ es v'alida, y solo permite comparar modelos anidados; la prueba  es un tanto conflictiva al comparar GTR con GTR+I o GTR+$\Gamma$; se asume que la forma como se agregan los par'ametros no sesga el resultado.\\
La otra forma est'a basada en medidas de informaci'on, como el criterio de Akaike, o la cantidad de informaci'on bayesiana. En estos casos se calcula c'uanta informaci'on contiene el modelo dados la cantidad de par'ametros que posee. La ventaja es que se puede hacer la comparaci'on entre modelos sin tener que tomar una secuencia particular.
\section{Materiales}
\noindent
Matriz de datos (datos.modelo.dat).
\section{M'etodos}
\noindent
(1) Abra la matriz de datos y ejecute el archivo \textit{modelblock3} en \Pname{PAUP*}.\\
(2) Use la salida producida por \Pname{PAUP*}: model.scores. como archivo de entrada para \Pname{Modeltest}.\\
(3) Siga paso a paso el test jer'arquico presentado. Intente realizar otro test jer'arquico comenzando con otro juego de par'ametros (note que la salida de \Pname{ModelTest} incluye los ajustes de los 56 modelos explorados).\\
(4) Calcule el valor de AIC para 4 diferentes modelos: JC, K80, GTR y uno de su elecci'on. El valor de AIC ser'a mejor cuanto m'as peque~no\footnote{AIC= -2lnL + 2N, donde lnL es el ajuste del modelo reportado por PAUP* y N el n'umero de par'ametros libres}.\\
(5) Para los mismos modelos y el escogido por el hLRT en \Pname{ModelTest}, calcule el criterio de informaci'on bayesiano\footnote{BIC=-2lnL + Nlnn, donde lnn, es el logaritmo natural de la longitud de la secuencia}. Compare sus resultados, tanto para AIC como para BIC con el de sus compa~neros, y determine c'ual es o son los mejores modelos para esos criterios. Ordene los modelos del mejor al peor. 

\subsection{Programas}
\noindent
En general se puede usar \Pname{Modeltest} o \Pname{MrModeltest}, que es una modificaci'on de \Pname{Modeltest} pero calcula un menor n'umero de modelos; en caso que de usar \Pname{MrModeltest}, tenga en mente que \textbf{no} ha evaluado todos los modelos.

\subsection{Comandos}
\noindent
Modeltest es un programa que se ejecuta a trav'es de la l'inea de comandos, o usando un archivo por lotes (archivo .bat) tal y como lo hizo en la pr\'actica de alineamiento con POY (vea la p\'agina ~\pageref{ch:alinear}); la secuencia b'asica de instrucciones desde la l'inea de comandos es:\\

\begin{math}
\Cmd{modeltest  < model.scores >salida.txt}
\end{math}\\ 

donde salida.txt es su archivo de salida con la terminaci'on texto (y en formato de texto).
\section{Preguntas}
\subsection{Pr'actica}
\noindent
?`Es el modelo escogido por el criterio de Akaike igual al escogido en el test jer'arquico? ?`Usted esperar'ia que lo fuesen?\\
Dada la exploraci'on en (3), ?`es igual el resultado para las dos exploraciones? ?`Usted esperar'ia que fuesen iguales 'o desiguales?\\ 
Para las listas obtenidas en (5), ?`son iguales las listas en AIC y BIC?
\subsection{Generales}
\noindent
?`C'omo se relaciona el concepto de caracteres hom'ologos usado en los primeros laboratorios, con el concepto de los modelos?\\
?`Prefer\'ia usted usar siempre el mismo modelo (por ejemplo, JC)? Argumente su respuesta

\section{Literatura recomendada}
\noindent
Posada \& Crandall, 2001 [presentan de manera completa c'omo seleccionar modelos bas'andose en la maximizaci'on del ajuste].
\chapter{M'axima verosimilitud}
\section*{Introducci'on}
\label{ch:likelihood}
Algunos autores han sugerido que la parsimonia es un modelo de evoluci'on que asume que la tasa de transformaciones es baja (por ejemplo, Swofford et al., 1996; Felsenstein, 2004). Independientemente de si el argumento es o no correcto (vea una posici'on en contra en Farris, 1983 y la argumentaci'on de Steel, 2002), estos autores sugieren que deben usarse modelos expl'icitos de evoluci'on como los que se vieron en la pr'actica \ref{ch:molecular} sobre selecci'on de modelos (ver p\'agina \pageref{ch:molecular}).\\
En este caso se estima el 'arbol (la hip'otesis filogen'etica) que maximice la probabilidad de los datos actuales dado el modelo evolutivo sugerido para las secuencias a analizar (aunque ha habido intentos de generalizar los modelos de evoluci'on a datos morfol'ogicos, Lewis, 2002).\\
Uno de los principales problemas con la estimaci'on de verosimilitud es que el valor depende de la longitud de las ramas. En la actualidad, se ignora al principio el valor de las diferentes ramas, y luego de encontrar un 'arbol espec'ifico, se calcula el mejor valor de verosimilitud para una rama a la vez. Swofford et al. (1996) ofrecen una explicaci'on extensa de c'omo se hacen los c'alculos relacionados con estos m'etodos.
\section{T'ecnicas}
Los m'etodos para buscar y seleccionar 'arboles usados en verosimilitud son b'asicamente los mismos usados en parsimonia: a un 'arbol inicial se lo mejora con permutaci'on de ramas. Existen diferentes formas de encontrar un 'arbol inicial en verosimilitud. Los sistem'aticos moleculares que usan modelos suelen empezar por un 'arbol de distancias (NJ: neighbor joining). El problema de este inicio es que para una configuraci'on particular, solo existe un 'arbol de NJ\footnote{tal y como lo hemos recalcado previamente para los an\'alisis de distancia el orden de entrada influye en el resultado, la topolog\'ia resultante puede cambiar si se cambia el orden de entrada a la matriz.}. Otra forma es la descomposici'on de estrella, donde se tiene una politom'ia basal y se trata de resolverla desde la ra'iz. Este m'etodo es muy lento, puesto que a diferencia de un 'arbol de Wagner, la magnitud del problema es grande desde el inicio. Una alternativa adicional es comenzar con un 'arbol al azar, pero estos suelen ser sub'optimos, con lo cual la fase de permutaci'on es muy lenta. Una 'ultima forma, es utilizar 'arboles de Wagner, pero basados en verosimilitud, o si se desea una respuesta m'as r'apida, un 'arbol obtenido por parsimonia y luego permutar las ramas usando verosimilitud.
\section{Materiales}
\noindent
Matriz de datos (datos.like.dat).
\section{M'etodos}
\noindent
(0) Antes que nada estime el mejor modelo para la matriz. Argumente qu'e criterio utiliz'o para seleccionar el modelo.\\
(1) Abra la matriz de datos, busque el 'arbol m'as parsimonioso.\\
(2) Permute las ramas de ese 'arbol, usando el modelo preferido en (0). Anote el valor de la verosimilitud (-lnL) reportado.\\
(3) Pruebe alternativamente los siguientes modelos: JC, HKY, F81 y TRM, usando como frecuencia para las bases la estimaci'on emp'irica y sin el par'ametro $\Gamma$. Reporte los valores de verosimilitud para cada modelo.\\
(4) Repita el ejercicio hecho en (3), pero use la estimaci'on emp'irica del par'ametro $\Gamma$ para HKY, F81 y TRM. Reporte los valores de verosimilitud para cada modelo.\\
(5) Repita el ejercicio hecho en (3), pero use invariantes. Reporte los valores de verosimilitud para cada modelo.\\
(6) Estime si los datos se ajustan o no al reloj molecular.
\subsection{Programas}
\noindent
\Pname{PAUP*}, \Pname{PHYLIP} o \Pname{PAML}; entre otros\footnote{puede consultar 
\href{}http://evolution.genetics.washington.edu/phylip/software.html}.
\subsection{Comandos}
La b'usqueda inicial en \Pname{PAUP*} h'agala bajo el criterio de parsimonia, y posteriormente pase al criterio de ML y haga una \textrm{peque~na} b'usqueda sobre los 'arboles producto del an'alisis con parsimonia; recuerde salvar los 'arboles de cada modelo (incluida la longitud de las ramas). Para las b'usquedas puede utilizar los mismos comandos que se usaron en parsimonia.\\
Con el comando \Cmd{set criterion=likelihood} se coloca a \Pname{PAUP} en modo de verosimilitud.\\
Con el comando \Cmd{Lset} usted puede modificar los diferentes par'ametros de los modelos (funci'on $\Gamma$, distribuci'on de bases, tipos de cambios, invariantes).\\ 
Puede consultar el \Cmd{model.block} que acompa~na a \Pname{ModelTest} (y usado en la pr'actica sobre obtenci'on del modelo 'optimo) para que vea c'omo se implementa cada uno de los modelos. Recuerde que \Pname{ModelTest} usa expl'icitamente los 56 modelos; tambi'en puede usar la salida de \Pname{ModelTest} para cambiar el modelo. Las instrucciones generadas por \Pname{ModelTest} pueden ser incluidas \textbf{dentro} del archivo de datos, para lo cual puede usar un editor de texto, \Pname{Mesquite} o \Pname{MacClade}, pero \textbf{no} \Pname{WinClada}.
\section{Preguntas}
\subsection{Pr'actica}
\noindent
Compar'e sus resultados con los de sus compa~neros. ?`C'ual fue el mejor valor de verosimilitud, independientemente del modelo?\\
?`C'omo considera los tiempos de ejecuci'on comparados con otros m'etodos como parsimonia? ?`Difieren los resultados con los de parsimonia?
\subsection{Generales}
\noindent
Recientemente se ha propuesto que se pueden usar modelos para el an'alisis morfol'ogico. Analice los puntos favorables o desfavorables e indique su punto de vista sobre el tema. ?`Qu'e implicaciones tiene el uso de modelos en el concepto de homolog'ia y car'acter presentado anteriormente?
\section{Literatura recomendada}
\noindent
Swofford et al., 1996 [Una descripci'on muy completa de la forma como se calculan las versomilitudes y los m'etodos de verosimilitud para encontrar 'arboles].\\
Lewis, 2001 [Introduce el uso de modelos en morfolog'ia].
\chapter{An\'alisis Bayesiano}
\section*{Introducci\'on}
Aunque tiene una larga tradici\'on en an\'alisis de probabilidades, el an\'alisis bayesiano es la forma m\'as reciente de generar filogenias.  El an\'alisis est\'a basado en el teorema de Bayes (1763), donde se asignan probabilidades posteriores (en este caso la probabilidad de que el nodo en el \'arbol sea correcto) a determinadas soluciones de los datos (en este caso hip\'otesis filogen\'eticas) partiendo de probabilidades asignadas \textit{a priori} y dependiendo de la frecuencia de recuperaci\'on de una soluci\'on, la cual es asimilada a la probabilidad posterior en el teorema de Bayes.\\
A pesar de que la asignaci\'on de las probabilidades  \textit{a priori} es un tema cr\'itico dentro del c\'alculo de las probabilidades posteriores y son dif\'iciles de defender, generalmente se asume
que todos los \'arboles son igualmente posibles (una inferencia dura, dado que algunas soluciones 
podr\'ian ser biol\'ogicamente imposibles, o algunos nodos pueden no estar soportados) y la verosimilitud de las soluciones se calcula bajo alg\'un modelo de evoluci\'on (los mismos modelos que se utilizan en m\'axima verosimilitud).\\
El an\'alisis bayesiano ofrece la ventaja de examinar casi todas las posibles soluciones que pueden ser obtenidas dados los datos, en tiempos considerablemente cortos con respecto a otras optimizaciones como parsimonia/ML, en las cuales las b\'usquedas heur\'isticas son la \'unica opci\'on cuando se trata de matrices datos con m\'ultiples 30 taxa. Aunque algunas de las ventajas del an\'alisis bayesiano pueden parecer atractivas, se debe tener en cuenta que tanto la asignaci\'on de los priores, la selecci\'on de modelos de evoluci\'on y la selecci\'on final de los \'arboles obliga a hacer afirmaciones muy fuertes acerca de los procesos. Por ejemplo, algunas de las afirmaciones que justifican el uso de MCMCMC como muestreo parecen no cumplirse (e.g., que las probabilidades de cada grupo sean equivalentes a las halladas en un \textit{bootstrap} param\'etrico), una implementaci\'on apropiada contin\'ua siendo tema de discusi\'on dentro del an\'alisis filogen\'etico. Otro argumento en contra del analisis bayesiano es que la consistencia estad\'istica, que argumentan los defensores de maxima verosimilitud, no parece ser aplicable a la estimaci\'on bayesiana (Goloboff \& Pol, 2005).
\\
\section*{T\'ecnicas}
Dado que el an\'alisis bayesiano debe explorar las posibles soluciones de los datos; esto es, calcular las probabilidades posteriores de todos los \'arboles, necesita m\'etodos de c\'alculo muy poderosos. Para resolver esta exigencia, el an\'alisis bayesiano utiliza una cadena de Markov-Montecarlo bajo el algoritmo de
Metropolis-Hastings (resumido todo como MCMCMC o MCMC). El algoritmo MCMC funciona b\'asicamente seleccionando y desechando 
soluciones (\'arboles filogen\'eticos), siguiendo una caminata aleatoria; 
inicialmente, al escoger un \'arbol determinado este es perturbado (modificado)
azarosamente, lo que genera un nuevo \'arbol, que es rechazado o aceptado dependiendo de su probabilidad, la cual se calcula por medio del algoritmo de Metropolis \& Hastings; si este \'arbol es aceptado, sufre perturbaciones adicionales. Uno de los problemas que enfrenta el m\'etodo de MCMC es que la frecuencia con que se encuentran  determinados \'arboles es dependiente de sus probabilidades posteriores, lo que genera que las cadenas queden estacionadas en determinadas zonas del espacio de soluciones.

El ajuste de los priores por omisi\'on es una distribuci\'on plana de \textit{Dirichlet} donde todos los valores son 1.0. Este ajuste es apropiado si se desean estimar los par\'ametros desde los datos sin asumir un conocimiento \textit{a priori} acerca de los valores de estos par\'ametros. Sin embargo,  \Pname{MrBayes} permite ajustar los valores de las frecuencias nucleot\'idicas como iguales; por ejemplo, en el caso de estar usando modelos como JC o SYM, se puede ajustar un prior espec\'ifico que ponga mas \'enfasis sobre la igualdad de las frecuencias nucleot\'idicas de la que est\'a por omisi\'on.

Un problema con \Pname{MrBayes} es que durante el muestreo no se examina el soporte de las ramas y dado que la respuesta es un consenso de la mayor\'ia, las ramas que no tienen soporte pueden aparecer con probabilidades posteriores altas (Goloboff \& Pol, 2005).

%Matriz de datos (datos.bayes.dat).\\
%Formato de gr\'aficas gnuplot.txt.

Para leer los datos \Pname{MrBayes} utiliza \Cmd{execute} y el nombre del archivo, al igual que {PAUP*}. \Pname{Mrbayes} reconoce los comandos con las tres o cuatro primeras letras, por ejemplo, puede abrir los datos s\'olo con \cmd{exe}.

Con \cmd{} \Pname{MrBayes} inicia la b\'usqueda en autom\'atico utilizando los ajustes que est\'en predefinidos. Cambie el n\'umero de cadenas \cmd{nchains}, el n\'umero de generaciones \cmd{ngen} y la temperatura \cmd{temp} seguiendo con el comando \cmd{mcmcp};
por ejemplo, \Cmd{mcmcp ngen=1000} para ajustar el n\'umero de generaciones a 1.000.\\
Para ajustar cada cu\'anto ser\'a muestreada la cadena, use \Cmd{mcmcp samplefreq}
Por omisi\'on, la cadena es muestreada, cada 100 generaciones. Este ser\'a tambi\'en el mismo n\'umero en que se reportar\'an los resultados en el archivo de salida.

Si  desea ajustar frecuencias iguales para modelos como  JC o SYM, use el comando \Cmd{prset statefreqpr=fixed (equal)}. 

Si desea ajustar un priori espec\'ifico que enfatice la igualdad de las frecuencias nucleot\'idicas,  puede usar \Cmd{prset statefreqpr=dirichlet (10,10,10,10)}, o un mayor \'enfasis a\'un con \Cmd{prset statefreqpr=dirichlet (100,100,100,100)}. Con \Cmd{showmodel} puede ver en cualquier momento c\'omo esta configurado el modelo de sustituci\'on nucleot\'idica a usar.

Con \Cmd{help} se obtiene una lista de los comandos disponibles en \Pname{MrBayes}; al igual que otros de los programas usados con \cmd{help comando}, se obtiene una informaci\'on de ayuda sobre este comando en particular. Con \cmd{help lset} se obtiene una lista similar al obtenido con help, pero al final una tabla muestra los ajustes actuales y predefinidos del modelo molecular. Para ver los ajustes predefinidos de las b\'usquedas use \cmd{help MCMC}. 


\begin{enumerate}

	\item A partir de las secuencias alineadas use \Pname{JModelTest} para evaluar c\'ual es el mejor modelo de evoluci\'on del gen. Debe revisar el modelo propuesto por \Pname{JModelTest} y comparar con las equivalencias para \Pname{MrBayes} en la p\'agina:\url{http://mrbayes.sourceforge.net/}

	\item 	Realice una corrida en \Pname{MrBayes} con los modelos JC y con el (los) modelos propuestos por \Pname{JModelTest}, inicialmente con pocas generaciones (1000), pocas cadenas (2) y la $"$temperatura$"$ por defecto.
	%los valores de los par\'ametros con \Pname{GnuPlot}. Existe un formato automatizado, gnplot.txt, revise el ap\'endice donde existe un ejemplo. Eval\'ue si las corridas llegaron al punto de saturaci\'on.\\

	\item \Pname{MrBayes} produce dos salidas, un archivo .p donde se imprimen los par\'ametros del modelo de evoluci\'on (este archivo est\'a delimitado por tabuladores y es f\'acilmente exportable a programas para gr\'aficas) y un archivo .t donde se reportan los \'arboles y las longitudes de las ramas. Grafique el archivo de salida .p con Tracer:

	\begin{enumerate}
		\item A partir de la gr\'afica decida c\'uantos \'arboles desea eliminar antes de hacer el consenso de la mayor\'ia como resumen.

		\pregunta{?`Cree que el consenso estricto de los \'arboles obtenidos servir\'ia? Argumente su respuesta.}

		\item Aumente el n\'umero de generaciones a 100.000 y repita el an\'alisis.
	\end{enumerate}	

	\item Repita los procesos al menos 3 veces, y compare las topolog\'ias resultantes y los valores del consenso de la mayor\'ia obtenidos para el an\'alisis bayesiano.

\end{enumerate}





\preguntaGral{
	\begin{itemize}
	\item ?`Podr\'ia dejar que \Pname{MrBayes} seleccione el mejor modelo?
	\item Se ha propuesto que en el an\'alisis bayesiano se sobreestiman las ramas sin soporte al generar topolog\'ias muy resueltas. ?`C\'omo solucionar\'ia usted tal problema?
	\enḑ{itemize}
}


\section*{Literatura recomendada}

Goloboff \& Pol, 2005 [Una cr\'itica muy completa al analisis bayesiano].\\
Huelsenbeck et al., 2002 [Una muy buena presentaci\'on del an\'alisis bayesiano por sus principales defensores].

\input{soporte.ml.tex}
\small
% 
\renewcommand{\chaptername}{}
\chapter{Bibliograf'ia recomendada}
\noindent
Antezana M. 2003. When being ''most likely'' is not enough: examining the performance of three uses of the parametric bootstrap in phylogenetics. \textit{Molecular Evolution} 56:198-222\\
Bremer K. 1994. Branch support and tree stability. \textit{Cladistics} 10:295-304.\\
Camin J.H. \& Sokal R.R. 1965. A method for deducing branching sequence in phylogeny. \textit{Evolution} 19:311-326.\\
Carpenter, J.M. 1988. Choosing among multiple equally parsimonious cladograms. \textit{Cladistics} 4:291-296.\\
Coddington J. \& Scharff N. 1994. Problems with zero-length branches. \textit{Cladistics} 10:415-423.\\
Dayoff N.O. 1969. Computer analysis of protein evolution. \textit{Scientific American} 221:87-95.\\
Farris J.S. 1970. Methods for computing Wagner trees. \textit{Systematic Zoology} 19:83-92.\\
Farris J.S., Kluge A.G. \& Eckardt M.J., 1970. On predictivity and efficiency. \textit{Systematic Zoology} 19:363-72.\\
Farris J.S., Albert V.A., Kallersjo M., Lipscomb D. \& Kluge A.G. 1996. Parsimony jackknifing outperforms neighbor-joining. \textit{Cladistics} 12:99-24.\\
Felsenstein J. 2004. \textit{Inferring phylogenies}. Sinauer.\\
Fitch W. M. 1971. Toward defining the course of evolution: minimum change for a specific tree topology. \textit{Systematic Zoology} 20:406-416.\\
Frost et al 2001\\
Grant T. \& Kluge A.G. 2004. Transformation series as an ideographic character concept. \textit{Cladistics} 20:23-31.\\
Giribet \& Wheeler 1999 \\
Goloboff P.A 1997. Self-weighted optimization: Tree searches and state reconstructions under implied transformation costs. \textit{Cladistics} 13:225-245.\\
Goloboff P.A. 1998. Tree searches under Sankoff parsimony. \textit{Cladistics} 14:229-237.\\
Goloboff P.A. 1999. Analyzing large data sets in reasonable times: solutions for composite optima. \textit{Cladistics} 15:415-428.\\
Goloboff P.A., Farris J.S., Kallersjo M., Oxelman B., Ram'irez M.J. \& Szumik C.A.  2003. Improvements to resampling measures of group support. \textit{Cladistics} 19:324-332.\\
Goloboff P.A. \& Farris J.S. 2001. Methods for quick consensus estimation. \textit{Cladistics} 17:S26-S34.\\
Goloboff P.A. \& Pol, D. 2005. Parsimony and Bayesian phylogenetics. En: \textit{Parsimony, Phylogeny, and Genomics} (Ed. V.A. Albert). Oxford, pp. 148-162.\\
Goloboff P.A., Mattoni C.I. \& Quinteros A.S. 2006. Continous characters analyzed as such. \textit{Cladistics} 22:589-601.\\
Hennig W. 1968. \textit{Fundamentos de una sistemtica filogen'etica}. Eudeba.\\
Huelsenbeck J.P., Larget B., Miller R.E. \& Ronquist, F. 2002. Potential applications and pitfalls of Bayesian inference of phylogeny. \textit{Systematic Biology} 51:673-688.\\
Kearney M. 2002. Fragmentary taxa, missing data, and ambiguity: mistaken assumptions and conclusions. \textit{Systematic Biology} 51:369-381.\\
Kluge A.G. \& Farris J.S. 1969. Quantitative phyletics and the evolution of anurans. \textit{Systematic Zoology} 18:1-32.\\
Kluge A.G. 2002. Distinguishing -or- from -and- and the case for historical identification. \textit{Cladistics} 18:585-593.\\
Kluge A.G. 2003. The repugnant and the mature in phylogenetic inference: atemporal similarity and historical identity. \textit{Cladistics} 19:356-368.\\ 
Lewis 2001 morfologia y ml\\
Maddison W.P. \& Maddison D.R. 1992. \textit{MacClade version 3: Analysis of phylogeny and character evolution}. Sinauer.\\
Maddison D.R., Swofford D.L. \& Maddison W.P. 1997. NEXUS: An extensible file format for systematic information. \textit{Systematic Biology}, 46:590-621.\\
Miranda-Esquivel D.R., Garz'on I. \& Arias J.S. 2004. ?`Taxonom'ia sin historia? Entom'ologo 32, 4-7.\\
Neff N.A. 1986. A rational basis for a priori character weigthing. \textit{Systematic Zoology} 35:110-123.\\
Nelson G. 1979. Cladistic analysis and synthesis: principles and definitions, with a historical note on Adanson- Familles des plants (1763-1764). \textit{Systematic Zoology} 28:1-21.\\
Nixon K.C. 1999. The parsimony ratchet, a new method for rapid parsimony analysis. \textit{Cladistics} 15:407-414.\\
Nixon K.C. \& Carpenter J.M. 1996. On consensus, collapsibility, and clade concordance. \textit{Cladistics} 12:305-321.\\
Lewis P. 2001. A likelihood approach to estimating phylogeny from discrete morphological character data. \textit{Systematic Biology} 50:913-925.\\
Page R.D. \& Holmes E.C. 1998. \textit{Molecular Evolution: A Phylogenetic Approach}. Blackwell Publishers.\\
Patterson C. 1981. Significance of fossils in determining evolutionary relationships. \textit{Annual Review of Ecology and Systematics} 12:195-223.\\
Patterson C. 1982. Morphological character and Homology. En: \textit{Problems of phylogenetic reconstruction} (Ed. K.A. Joysey, A.E. Friday) . Academic, pp. 21-74.\\
de Pinna M.C.C. 1991. Concepts and tests of homology in the cladistic paradigm. \textit{Cladistics} 7:367-394.\\
Platnick N.I. 1979. Philosophy and the transformation of cladistics. \textit{Systematic Zoology} 28: 537-546.\\
Pleijel F. 1995. On character coding for phylogeny reconstruction. \textit{Cladistics} 11:309-315.\\
Posada D. \& Crandall K.A. 2001. Selecting the best-fit model of nucleotide substitution. \textit{Systematic Biology} 50:580-601.\\
Quicke D.L.J, Taylor J. \& Purvis A. 2001 Changing the landscape: a new strategy for estimating large phylogenies. \textit{Systematic Biology} 50:60-66.\\
Ram'irez M.J. 2005. Resampling measures of group support: a reply to Grant and Kluge. \textit{Cladistics} 21:83-89.\\
Rice K.A., Donoghue M.J. \& Olmstead R.G. 1997. Analyzing large data sets: rbcL 500 revisited. \textit{Systematic Biology} 46:554-563.\\
Richards R. 2002. Kuhnian valures and cladistic parsimony. \textit{Perspectives on Science} 10:1-27.\\
Richards R. 2003. Character individuation in phylogenetic inference. \textit{Philosophy of Science} 70:264-279.\\
Rieppel O. \& Kearny M. 2002. Similarity. \textit{Biological Journal of the Linnean Society} 75:59-82.\\ 
Scotland R. W., Olmstead R.G. \& Bennett J.R. 2003. Phylogeny reconstruction: the role of morphology. \textit{Systematic Biology} 52:539-548.\\
Sharkey M.J. \& Leathers J.W. 2001. Majority does not rule: the trouble with majority-rule consensus trees. \textit{Cladistics} 17:282-284.\\
Sokal R.R. 1983. A phylogenetic analysis of the Caminalcules. I. The data base. \textit{Systematic Zoology} 32(2): 159-184.\\
Steel M. 2002. Some statistical aspects of the maximum parsimony method. En: \textit{Molecular Systematics and Evolution: Theory and Practice} (Ed. R. DeSalle, G. Giribet \& W. Wheeler). Birkhauser, pp. 125-140.\\
Strong, E.E. \& Lipscomb, D. 1999. Character codding and inaplicable data. \textit{Cladistics} 15:363-371.\\
Swofford, D. 1991. When are phylogeny estimates from molecular and morphological data incongruent? En: \textit{Phylogenetic Analysis of DNA Sequences} (Ed. M.M. Miyamoto, J. Cracraft). Oxford, pp. 295-333.\\
swofford et al 1996 format nexus\\
Swofford D.L., Olsen G.J., Waddell P.J. \& Hillis D.M. 1996. Phylogenetic inference. En: \textit{Molecular systematics} (Ed. D.M. Hillis, C. Moritz, B.K. Mable). Sinauer, pp. 407-514.\\
Tuffley C. \& Steel M. 1997. Links between maximum likelihood and maximum parsimony under a simple model of site substitutions. \textit{Bull. Math. Biol}. 59, 581-607.\\
Wagner W.H. 1961. Problems in classification of ferns. En: \textit{Recent advances in botany}, vol. 1. Toronto, pp. 841-844.\\
Wiley E.O. 1981. \textit{Phylogenetics}. John Wiley and Sons.\\
Wheeler 1995 y et al 2006
% \backmatter
\appendix
\small
\chapter{Programas de c'omputo}
\section*{}
\label{ch:programas}

Hoy en d'ia los an'alisis filogen'eticos usan decenas, cientos a 
miles de terminales y de caracteres, por lo que se requiere gran 
cantidad de c'alculos. Aunque es posible hacerlos a mano, esto se 
hace imposible para m'as de 10 taxa y una veintena de caracteres. 
Esta secci'on busca familiarizar al estudiante con los programas 
m'as comunes, aparte de los usados durante las distintas pr'acticas. 
Es casi un truismo que la selecci'on de programas depende de la(s) 
plataforma(s) utilizada(s), el problema a resolver, y en menor grado 
de los fondos disponibles. Dado que ''por unos pocos d'olares'' se 
puede tener un repertorio apropiado de programas, la condici'on 
central a tener en cuenta es la \textbf{velocidad}. 
La \textbf{facilidad de manejo} en la gran mayor'ia de casos es secundaria; 
aunque la curva de aprendizaje puede ser lenta y tortuosa, las 
habilidades ganadas hacen que los an'alisis en el largo plazo sean 
m'as faciles de ejecutar; a eso se suma la posibilidad de manejar 
''lenguajes'' por casi todos los programas. Es posible que al final 
del aprendizaje todo sea asunto de un par de instrucciones, una vez 
que ha perfeccionado sus habilidades y que posee archivos de 
instrucciones preajustadas a sus gustos y necesidades.

\section{Editores de matrices y manejadores de b'usqueda}
\index{Programas!Edici\'on de matrices}
\subsection{WinClada}
\noindent
Autor: Nixon, 2002.\\
Plataforma: Windows 9x o superior. [Funciona MUY bien en \Pname{wine} dentro de Linux].\\
Disponibilidad: Shareware (30 d\'ias de prueba), tiene un costo de US 50.00.\\
http://www.cladistics.com
% \href{http://www.cladistics.com}.
\\El programa require de \Pname{NONA} para realizar las b'usquedas (\Pname{PIWE}/\Pname{Hennig86} son deseables, pero no obigatorios).
\paragraph*{}
Bajo Windows, este es uno de los programas m'as 'utiles, tanto para principiantes como para iniciados. El programa tiene dos entornos para manejar matrices y 'arboles. El manejo e impresi'on de 'arboles permite que se tengan im'agenes listas para publicaci'on con pocos pasos y sin necesidad de retoque posterior. Tiene un par de ''conflictos'' con el manejo de im'agenes que pueden ser inc'omodos; adem'as, el usuario debe tener presente que el programa puede numerar (taxa, caracteres y 'arboles) desde cero (la manera l'ogica para los programadores), o desde uno, por lo que se debe tener cuidado al escribir el texto y referenciar la gr'afica.
\subparagraph*{WinDada}
\Gui{Matrix}: Le permite cambiar algunos aspectos de la matriz (crearlas, cambiar tama~no, salvarlas, fusionarlas).\\
\Gui{Edit}: Lo m'as importante es que permite que los datos sean o no modificables. Adem'as, permite adicionar polimorfismos.\\
\Gui{Terms/Chars}: Estos men'u le permiten modificar cosas espec'ificas de terminales y caracteres, tales como nombre y orden, seleccionarlos, borrarlos, adicionar.\\
\Gui{View}: Opciones de presentaci'on. Adem'as, puede ver estad'isticos de los caracteres, aditivos, activos, pasos m'inimos y m'aximos; y le permite intercambiar modo num'erico o IUPAC (para DNA).\\
\Gui{Output}: Le permite exportar en formatos diferentes, o informaci'on variada sobre los caracteres.\\
\Gui{Key}: Si tiene los caracteres nominados, funciona como clave interactiva.\\
\Gui{Analize}: Tiene diferentes entradas para b'usquedas y c'alculo de soporte. En \Gui{heuristics}, es posible configurar la cantidad m'axima de 'arboles a retener, la cantidad de 'arboles por cada r'eplica ('arbol de Wagner+TBR) (i.e. estrategias NONA/PAUP) y otros detalles de la b'usqueda, como especificar una semilla para los n'umeros aleatorios (0 es el tiempo interno de la computadora). En \Gui{ratchet}: se puede modificar la cantidad de iteraciones de ratchet, el n'umero de 'arboles a retener por iteraci'on y el n'umero de b'usquedas de ratchet, bien sean secuenciales o simult'aneas. Mientras que con \Gui{bootstrap/jackknife/CR with NONA} permite manipular las diferentes opciones para hacer soporte con remuestreo. Recuerde borrar todos los 'arboles antes de ejecutar la b'usqueda de remuestreo.
\subparagraph*{CPanel}
Para llenar la matriz.\\
\Gui{Mode}: Cambia el modo de acceso de los caracteres.
\subparagraph*{Winclados}
Para manejar el 'arbol. En general, la mayor parte de estas acciones son accesibles desde la barra de herramientas con el rat'on.\\
Las teclas F2-F12 tienen (con y sin SHIFT) diversas funcionalidades de visualizaci'on, como engrosar/adelgazar el ancho de las ramas, expandir/contraer la longitud, moverse entre 'arboles.\\
\Gui{Trees}: opciones para visualizar y manipular el 'arbol sin modificar su topolog'ia (forma, consensos, ir a un 'arbol determinado).\\
\Gui{Nodes}: Para ver frecuencias de nodos, desafortunadamente su funcionalidad es muy inestable, y a veces produce resultados inesperados.\\
\Gui{HashMarks}: Para ver las transformaciones en los nodos.\\
\Gui{Edit/Mouse modes:} Modifica las acciones posibles del rat'on para modificar el 'arbol, como mover nodos, colapsarlos, cambiar la ra'iz.\\
\Gui{Diagnoser}: Para mapear caracteres.

\subsection{MacClade \& Mesquite}
\noindent
Autores: Maddison \& Maddison, 2005.\\
Plataforma (\Pname{McClade}): Macintosh (MacOS X, PPC y Classic 68k. Esta 'ultima funciona bien con emuladores como BasiliskII)\\
Plataforma (\Pname{Mesquite}): Cualquiera, requiera Java virtual machine.\\
Disponibilidad (\Pname{McClade}): Descontinuado pero puede acceder desde el sitio de los autores para el programa que funciona en OS X (hasta 10.6):\\
http://macclade.org/index.html.
\\
Disponibilidad (\Pname{Mesquite}): Gratuito.\\ http://mesquiteproject.org.
\\
\paragraph*{}
\Pname{MacClade} es un programa cuyas cualidades gr'aficas son impresionantes, su interfase es agradable, sencilla y no transmite miedo a los novatos. Dadas sus magnificas propiedades gr'aficas, es \textbf{muy} recomendable para la edici'on de 'arboles y matrices, pero principalmente para la optimizaci'on de caracteres, la cual permite distintos tipos de cambio de los mismos (v.g., Dollo, ACTRAN, DELTRAN) y adicionalmente tiene una caracter'istica 'unica de mapeo, el ''equivocal ciclying'' [Maddison \& Maddison, 1992].\\
En \Pname{MacClade} la edici'on de los datos y el manejo de los 'arboles se llevan a cabo en dos interfaces separadas para 'arboles y matrices, las cuales son accesibles en la ventana \Gui{Windows}. Dado que Macclade esta dise\~nado para ser igualmente funcional con datos morfol'ogicos y moleculares,  existen m'ultiples herramientas de edici'on.\\
\subparagraph*{Data Editor}
En esta interfaz se construye y edita la matriz. Sus ventanas son:\\
\Gui{Edit}: se pueden duplicar caracteres o taxa, editar bloques de comandos que corran directamente en PAUP. \\
\Gui{Utilities}: se pueden buscar secuencias particulares dentro de la secuencia total, reemplazar caracteres por otros, reemplazar
datos ausentes por \textit{gaps} y viceversa, importar alineamientos desde el \textit{genbank} y finalmente se puede lograr que la matriz hable por si sola !`s'i, que hable! Un lector autom'atico lee en forma descendente los taxa de su matriz facilit'andote un poco el trabajo (puede usar de hecho ingl'es brit'anico, si esto lo hace m'as feliz).\\
\Gui{Characters}: permite adicionar caracteres, incluir los estados, ver una lista de los caracteres que se han incluido, determinar el formato de los caracteres que est'an en la matriz (si son prote'inas, nucle'otidos, o est'andar, que es el definido para simbolog'ia de n'umeros en las casillas); tambi'en permite determinar el tipo de cambio que se permitir'a para los caracteres (cuando se est'a haciendo una optimizaci'on), el peso de los caracteres.\\
\Gui{Taxa}: permite hacer cosas semejantes a \Gui{characters}, pero para el manejo de los taxa, incluir taxa nuevos, crear listas de taxa o reordenarlos. \\
\Gui{Display}: finalmente esta ventana maneja toda la configuraci'on gr'afica de la matriz, el tipo de letra, su tama\~no, el color de los caracteres, el ancho de columna, etc.\\
La \'ultima ventana en el men'u es \Gui{Windows}, el cual permite moverse entre las interfaces de los datos y el 'arbol y llamar a una caja de herramientas que por omisi'on siempre est'a presente en la parte inferior izquierda de la ventana. Esta caja ofrece herramientas como llenar estados, expandir columnas, cortar, entre otras. En esta ventana \Gui{notes about trees} y \Gui{note file} permiten incluir comentarios acerca de los 'arboles y sobre la matriz.
\subparagraph*{Tree Window}
Esta interfaz maneja y modifica los 'arboles y posee las ventanas b'asicas del \Gui{data editor}, no obstante, incluye otras nuevas, espec'ificas para el mapeo de caracteres y manejo de los 'arboles. Solo se describen aquellas nuevas ventanas: \\
\Gui{Trees}: Esta ventana permite cambiar, abrir archivos externos para incluir 'arboles, crear una lista de los 'arboles que esta incluidos a la matriz, guardar los 'arboles, exportarlos en formato de hennig86 o Phylip (ver anexo de formatos para m'as informaci'on) y manejar las politomias como politomias blandas o duras (Coddington \& Scharff, 1994).\\
\Gui{$\Sigma$}: esta ventana incluye el manejo de todos los estad'isticos  del 'arbol, como la longitud, el 'indice de consistencia, 'indice de retenci'on, 'indice de consistencia escalonado, el n'umero de cambios en el 'arbol y finalmente, puede generarse el archivo de comandos para correr el indice \Gui{decay}, 'o soporte de Bremer en \Pname{PAUP*} (vea el cap'itulo sobre programas).\\
% aqui debe ponerse una etiqueta dirigida a programas paup
\Gui{Trace}: aqu'i se maneja todo lo relacionado con el mapeo de los caracteres; puede mapear todos los caracteres a la vez, o escoger un determinado car'acter para ser mapeado. Tambi'en se escoge el tipo de cambio de los caracteres (\Gui{resolving options}) el cual puede ser ACTRAN o DELTRAN, y escoger el tipo de mapeo para los caracteres continuos, \Pname{MacClade} y \Pname{Mesquite} son los 'unicos programas que permiten mapear tal tipo de caracteres.\\
\Gui{Chart}: En esta ventana se pueden generar cuadros de estad'isticas de los caracteres vs pasos y 'arboles (characters states/etc.) o de los cambios de caracteres, la ocurrencia de los estados  a trav'es de todos los caracteres. Finalmente puede comparar dos 'arbolesresolviendo sus politomias o tal y como son.\\
\Gui{Display}: finalmente en esta ventana se manejan los aspectos gr'aficos de la resentaci'on de los 'arboles tales como el tipo de letra, el tama\~no y el estilo de la letra, el estilo y forma del 'arbol, numeraci'on de las ramas, el tama\~no del 'arbol, la configuraci'on de los colores  del mapeo y la simbolog'ia del los taxa (como n'umeros o nombres).\\
\\
Aunque con menos capacidades que \Pname{McClade}, \Pname{Mesquite} es una herramienta poderosa, teniendo en cuenta que es gratuito. Su principal falla es que no permite ver las transformaciones en los nodos de todos los caracteres simult'aneamente. Su interfaz es muy similar a la de \Pname{McClade} y la distribuci'on de ventanas y comandos es m'as o menos equivalente.\\

\section{B'usqueda de 'arboles}
\index{Programas!B\'usquedas}
\subsection{TNT}
\noindent
Autor: Goloboff et al., 2007.\\
Plataforma: MS-DOS, Windows, MacOS, Unix.\\
Disponibilidad: es posible descargar un demo funcional, para 10 corridas. 
El programa tiene un costo de US80.00.\\
\href{}http://www.zmuc.dk/public/phylogeny/TNT/.
\\
Para parsimonia, es el programa m'as r'apido que se ha desarrollado; incluye varios tipos de b'usquedas especializadas, como la deriva y la fusi'on de 'arboles.\\
\\
Al igual que \Pname{NONA}, \Cmd{proc} le permite abrir la matriz de datos o archivos de instrucciones. Para las b'usquedas puede configurar los diferentes m'etodos usando \Cmd{ratchet}, \Cmd{drift} y \Cmd{mult}; al usar \Cmd{?} puede ver c'uales son los parametros, y con \Cmd{=} puede modificarlos.\\ \Cmd{Mult} puede usarse como la ''central de b'usqueda'', definiendo un n'umero de r'eplicas para una b'usqueda de Wagner y las posteriores mejoras con ratchet y drift (seg'un como est'en configurados). Para correr simplemente escriba \Cmd{mult: replic X;} para ejecutar el n'umero de replicas que desee (X).\\
Con \Cmd{tsave* nombre} se abre el archivo ''nombre'' para guardar 'arboles. Gu'ardelos con \Cmd{save} y al final no olvide cerrar el archivo: \Cmd{tsave/;}\\
El programa cuenta con ayuda en l'inea que puede ser consultada usando \Cmd{help} o escribiendo \Cmd{help comando} para un comando espec'ifico.\\
Este es 'unico programa que tiene directamente implementado el remuestreo sim'etrico. Adem'as tiene implementado el soporte de Bremer, el soporte relativo de Bremer, el soporte de Brtemer dentro de los l'imites del Bremer absoluto y puede calcular la cantidad de grupos soportados-contradichos (FC). Con \Cmd{resample} usted puede configurar como quiere la permutaci'on. Con \Cmd{subop} (tambi'en en \Pname{NONA}) usted puede indicar la longitud de los sub'optimos, en diferencia de pasos con respecto al 'arbol 'optimo. \Cmd{Bsupport} hace el c'alculo del 'indice de Bremer, con un \Cmd{*} calcula el valor relativo, o puede usar \Cmd{Bsupport ]} para calcular el soporte relativo dentro de los l'imites del absoluto. Infortunadamente los resultados no se almacenan en ninguna parte, por lo que debe generar una salida (en formato de texto).\\
\subparagraph*{Versi'on de men'u}
Solo para Windows es bastante similar a \Pname{winClada}, pero esta mucho mejor organizada. Algunas funciones son:\\
\Gui{Settings}: Adem'as de las diferentes opciones de macros, manejo de memoria, tambi'en contiene los par'ametros usados en el colapsado de ramas, consensos, y pesado impl'icito.\\
\Gui{Analyze}: Contiene los diferentes tipos de b'usquedas y sus par'ametros, el manejo de 'arboles suboptimos, y los remuestreos.\\
\Gui{Optimize}: Permite ver y dibujar sinapomorf\'ias, mapear caracteres, y revisar estad\'isticos de caracteres y 'arboles (por ejemplo long o peso).\\
\Gui{Trees}: All'i se encuentran las entradas para el dibujo de arboles, el calculo soporte de Bremer, realizar consensos y super-'arboles, as'i como 'arboles al azar, y el manejador de etiquetas de los nodos (\textit{tags}).\\
\Gui{Data}: Aparte de la configuraci\'on�de caracteres y terminales, posee un editor b'asico de datos, que aunque muy simple, es extremadamente f'acil de manejar, y mas directo que los editores basados en mostrar la matriz (Como \Pname{winClada}, o \Pname{Mesquite}).\\
Usted puede encontrar un manual para el lenguaje de macros en la direcci\'on:\\ 
\href{}http://www.zmuc.dk/public/phylogeny/TNT/scripts/
\subsection{NONA}
\noindent
Autor: Goloboff, 1998.\\
Plataforma: Linux, Mac o Windows (9x o superior).
Disponiblilidad: Gratuito en conjunto con otros programas en\\
http://www.zmuc.dk/public/phylogeny/Nona-PeeWee.
% \href{http://www.zmuc.dk/public/phylogeny/Nona-PeeWe}.
\\
\paragraph*{}
Este programa es la mejor opci'on que existe para b'usquedas bajo parsimonia, dado que es gratuito, tiene lenguaje de macros y es veloz. Si est'a interesado en b'usquedas usando concavidad, puede usar \Pname{PIWE}; si el inter'es son las b'usquedas de Sankoff entonces \Pname{PHAST}/\Pname{SPA} son los programas requeridos. Todos estos son similares a \Pname{NONA} en cuanto a comandos se refiere, y est'an incluidos con la distribuci'on de \Pname{NONA}.
\paragraph*{}
B'asicamente los comandos de \Pname{NONA} son los mismos de \Pname{TNT}. Una diferencia importante (aparte de la velocidad y algoritmos implementados) es que en \Pname{NONA} los comandos son ejecutados (en vez de poder configurar sin ejecutar). A diferencia de \Pname{PAUP*}, no puede desactivar terminales.\\
Al igual que en \Pname{TNT}, el comando simple de b'usqueda es \Cmd{mult}. Para permutar ramas se utiliza \Cmd{max}. En ambos casos debe colocarse un asterisco \Cmd{mult*; max*} para que utilice TBR como permutaci'on, en caso contrario usar\'a SPR.\\
Para salvar el 'arbol, el comando es \Cmd{sv}. La primera vez que se llama, solo abre el archivo (Debe darse como par'ametro o el programa solicita un nombre). Con \Cmd{sv*} salva todos los 'arboles en memoria. Usando \Cmd{ksv} se guardan los 'arboles colapsados, de lo contrario siempre se guardan dicot'omicos. Para salvar el consenso es necesario usar \Cmd{inters}.\\
El 'unico m'etodo de soporte expl'icitamente implementado es el soporte de Bremer \Cmd{bsupport}: con asterisco \Cmd{bs*} muestra los soportes relativos, 'o de lo contrario muestra el valor absoluto. Pero el programa viene acompa\~nado de varios programas y macros que permiten calcular \textit{jackknife} y \textit{bootstrap}, as'i como hacer medidas basadas en frecuencias relativas (FC).

\subsection{Poy}
\noindent
Autor: Var'on et al., 2007.\\
Plataforma: Linux, MacOS y Windows.\\
Disponibilidad: gratuito.\\
\href{}http://research.amnh.org/scicomp/projects/poy.php
\\
Es el 'unico programa disponible que realiza un an'alisis simult\'aneo sin utilizar alineamiento previo de las secuencias moleculares. Tambi'en puede usarse para realizar alineamientos, o para hacer b'usquedas convencionales de datos previamente alineados o morfol'ogicos. La versi'on actual posee 'unicamente una implementaci'on de parsimonia, pero se planean versiones que tambi'en realicen b'usquedas bajo m'axima verosimilitud.\\
\\
El programa posee cuatro ventanas: una de salida, donde se imprimen salidas pedidas por el usuario (por ejemplo, un 'arbol, o la ayuda). Una ventana de comandos, donde se escriben las ordenes al programa, una donde muestra que clase de tarea se esta realizando y en la \'ultima donde se muestra el progreso de las b'usquedas.\\
Las matrices y arboles son abiertos usando \Cmd{read(''archivo'')}, donde archivo es el archivo a leer. Puede leer multiples conjuntos de datos, bien sea abriendo uno por uno, o varios en la misma instrucci'on como \Cmd{read(''arch1'',''arch2'')}. Los datos se van agregando. Para limpiar la memoria se utiliza \Cmd{wipe()}.\\
Se buscan 'arboles con \Cmd{build(x)}, que realiza x arboles de Wagner. Para mejorar dichos \'arboles hay que usar \Cmd{swap()}. Notese que las instrucciones est'an separadas. Es posible hacer b'usquedas m'as sofisticadas usando nuevas tecnolog'ias, como ratchet implementado en \Cmd{perturb(iterations: x, ratchet())}, donde se har'ian x iteraciones de rathcet, con \Cmd{swap(drift)} o \Cmd{swap(annealing)} se hace deriva de 'arboles y cristalizaci'on simulada, y \Cmd{fuse()} hace fusi'on de 'arboles.\\
El comando para manejar costos es \Cmd{transform(tcm())}. Se puede modificar los costos de transformacion para datos ''est'aticos'' como morfolog'ia con 

\Cmd{transform(static, weight:2)} 

que dar'ia un peso de 2 a cada transformaci'on de datos morfol'ogicos. Con 

\Cmd{transform(tcm(1,2))} 

se da un peso de 1 a las sustituciones y de 2 a los gaps y caracteres est'aticos, ese es el valor que viene por omisi'on. Con 

\Cmd{transform(fixedstates)} 

se implementa el m'etodo de estados fijos de Wheeler.


Es posible calcular soportes con 

\Cmd{calculate\_support()} 

que incluye \textit{bootstrap}, \textit{jackknife} y soporte de Bremer (por omisi'on). Las salidas se realizan con 

\Cmd{report(''archivo'')} 

 pueden ser cladogramas en formato parentical, dibujados en ascii, o en \textit{postscript}. Tambi'en es 'util para conocer estad'isticas de los 'arboles, las optimaciones de los nodos y la integridad de los datos.
 
 
El programa funciona desde computadoras de escritorio hasta \textit{clusters}. Puede ejecutarse directamente en la l\'inea de comando o en procesos de lotes.

\subsection{PAUP*}
\noindent
Autor: Swofford, 2002.\\
Plataforma: MWindows, MacOS, Unix.\\
Disponibilidad: el programa para l interface gr\'afica tiene un costo entre US 80.00 a US150.00 dependiendo del sistema operativo y es distribuido por Sinauer, mientras que para l\'inea de comandos es gratuito.\\
http://www.sinauer.com/detail.php?id=8060.


\paragraph*{}
\Pname{PAUP*} es uno de los programas m\'as usados en cuanto a b\'usquedas mediante ML, o si el sistema operativo es Mac. Dados los costos es muy posible que la interface dominante llegue a ser por l\'inea de comandos; la interface para Windows y MacOSX en modo gr\'afico tiene el mismo acercamiento; como \Pname{NONA} y \Pname{TNT} tiene la opci\'on de manejar un gran n\'umero de parametros que permiten hacer una b\'usqueda \textbf{sobre pedido}, adicionalmente realiza todos los consensos directamente sin necesitar un segundo programa. Su gran desventaja es su velocidad y la falta de un lenguaje de macros.


\cmd{set}:esta funci\'on define la configuraci\'on b\'asica de algunos par\'ametros. Son importantes \cmd{increase=no}, para que no le pregunte si desea aumentar el n\'umero de \'arboles, y \cmd{maxtrees=N} para dar como opci\'on que guarde un n\'umero de \'arboles (N).\\
\cmd{exec}: para abrir la matriz de datos o archivos de instrucciones.\\
\cmd{hsearch}: es la funci\'on de b\'usqueda de cladogramas. Las opciones de este comando son \cmd{addseq=random}, lo que asegura que la entrada de datos para el \'arbol de Wagner es al azar.\\
\cmd{swap=tbr/spr}: es el tipo de permutaci\'on.\\
\cmd{nreps=X}: indica el n\'umero de r\'eplicas que se har\'a en la b\'usqueda.\\ 
\cmd{nchuck=X ckuckscore=1}: le permite usar la estrategia NONA, indicando c\'uantos \'arboles desea por r\'eplica. Si usted desea hacer solo permutaci\'on de ramas del \'arbol previamente construido, la serie de instrucciones ser\'ia \cmd{search start=current chuckscore=no}.\\
Para guardar los \'arboles utilice el comando \cmd{savetre}.\\
Con el comando \cmd{contree} se generan los \'arboles de consenso. Estos deben ser guardados cuando se ejecuta el comado (par\'ametro \cmd{treefile=nombre}, donde nombre es el achivo.). Los m\'etodos de medici\'on de soporte implementados son el \textit{bootstrap} con el comando \cmd{bootsptrap} y el \textit{jackknife} con \cmd{jackknife}.\\
En caso de dudas, el programa posee una ayuda en linea, cons\'ultela usando \cmd{help} o escribiendo \cmd{help comando} para un comando espec\'ifico; para ver las opciones del comando use \cmd{comando ?}.
\subsection{MrBayes}
\noindent
Autor: Ronquist et al., 2005.\\
Plataforma: Windows, MacOS, Unix.\\
Disponibilidad: gratuito.\\
http://mrbayes.net
\\
Es el programa m'as com'unmente usado para an'alisis bayesiano. Es gratuito y su interfaz es muy similar a la de PAUP*. Los principales comandos del programa son descritos en el capitulo de an'alisis bayesiano. El tutorial incluido es muy completo y viene incluido dentro del programa.\\
Una de las caracter\'isticas m\'as interesantes es la implementaci\'on de los ''modelos'' usados para caracteres morfol\'ogicos (Lewis, 2001), as\'i como el modelo de 'parsimonia' tal y como fue descrito por Tuffley \& Steel (1997) que se activa usando 

\Cmd {lset parsmodel=yes;}\\





%\section{Miscel'aneos}
%\subsection{Component 2.0}
\noindent
Autor: Page, 1994. \\
Plataforma: Windows 16 bits (puede tener incompatibilidad con Windows XP, ME y superiores; instale soporte para 16 bits si desea correrlo); hay una versi'on para Mac, pero con limitaciones.  (Funciona \textit{relativamente} bien en \Pname{wine} dentro de Linux)\\
Disponibilidad: el programa es gratuito y es distribuido desde la p'agina del autor; incluye el c'odigo fuente en Pascal\\
http://taxonomy.zoology.gla.ac.uk/rod/cpw.html
\\
Este programa es 'util para c'alculos de consensos; su principal aplicaci'on est'a en biogeograf'ia.
%\subsection{ModelTest}
\noindent
Autores: Posada \& Crandall, 2001. La 'ultima vers'ion es Posada 2005.
\\Plataforma: Cualquiera (C'odigo fuente en C).\\
Disponibilidad: Gratuito.\\
http://darwin.uvigo.es/people/dposada.html
\\
El programa utiliza como base los resultados de \Pname{PAUP*}, basado en una estimaci'on de la verosimilitud de un 'arbol producido por distancia de JC. El programa requiere un archivo de instrucciones para \Pname{PAUP*} distribuido con el programa, el cual debe ser ejecutado primero. Una vez se ha producido la salida de \Pname{PAUP*} (model.scores), esta se usa como base para los c'alculos de los par'ametros para \Pname{ModelTest}. La salida puede ser le'ida en un procesador de texto, donde se presentan los resultados de cada uno de los modelos examinados, tanto del test jer'arquico (hLRT) como del crit'erio de informaci'on de Akaike (AIC). Se pueden modificar mediante l'inea de comandos las intrucciones dadas a Modeltest para el c'alculo del modelo; de hecho, puede usar el programa como calculadora para $\chi ^2$.\\
Para plataforma Windows existe un manejador que le permite realizar todo el proceso con ayuda del rat'on, y para MacOSX el programa mismo tiene una peque~na interfaz.
%\input{}



\chapter{Formatos de archivos}
\section{Formatos de matrices}
\index{Formatos}
\index{Formatos!Matrices}

\subsection{NONA}
\noindent
Es v'alido para \Pname{NONA}, \Pname{TNT}, \Pname{Hennig86} (formato de morfolog'ia de \Pname{POY}) y \Pname{WinClada}; comienza con \Cmd{xread}, luego el n'umero de caracteres y el n'umero de taxa, los polim\'orficos entre par\'entesis angulares y los desconocidos con \Cmd{-} o \Cmd{?}. Al final, un punto y coma y si se desea la aditividad de los caracteres (comenzando desde 0). Termina con \Cmd{p/} o \Cmd{p-}, que los programas interpretan como fin del archivo.\\
\\
\noindent
\Cmd{xread 'Matriz ejemplo' 5 5\\
out        00000\\
alpha     10-20\\
beta      1102[01]\\
gamma  1?111\\
lamda    11111\\
;\\
cc -0.2 +3 -4;\\
p/;}\\
\\
Para ADN (en \Pname{NONA}) se usa \Cmd{dread}, con la clausula \Cmd{gap} seguida de \Cmd{?} si se quieren asumir los \textit{gaps} como desconocidos, o con \Cmd{;} si quiere que sean un quinto estado. Se usa codificaci'on tipo IUPAC.\\
\\
\Cmd{dread gap ; match . 'DNA' 5 5\\
out        ACGTC\\
alpha     AT-CG\\
beta      RTAAC\\
gamma CGAY-\\
lamda   TCNCC\\
;\\
cc -.;\\
p/;}

\subsection{PAUP*}
\noindent
Se inicia con la clausula \Cmd{\#nexus}, y luego con el bloque \Cmd{data}. se puede definir si los caractertes son morfol'ogicos, ADN o prote'inas. Los polim'orficos se colocan entre par'entesis redondos. ADN en formato IUPAC.\\
\\
\noindent
\Cmd{\#nexus\\
begin data;\\
dimensions ntax=4 nchar=5;\\
format missing=? gap=- symbols="0 1 2";\\
matrix\\
out          00000\\
alpha       10-20\\
beta        1102(01)\\
gamma    1?111\\
lamda      11111\\
;\\
end;\\
begin assumptions;\\
typeset \*tipoUno=unord:1-3 5, ord:4;\\
end;
\\
begin paup;}\\
\Cmd{
$[$Aqui puede colocar instrucciones espec'ificas de paup, por ejemplo b'usquedas$]$\\
}
\Cmd{hsearch add=random;\\
end;\\
\\
\#nexus\\
begin data;\\
dimensions ntax=4 nchar=5;\\
format missing=? gap=- datatype=dna;\\
matrix\\
out           ACGTC\\
alpha      AT-CG\\
beta  RTAAC\\
gamma    CGAY-\\
lamda      TCNCC\\
;\\
end;
}
\subsection{MALIGN}
\noindent
El esquema de \Pname{MALIGN} es similar al de GenBank.\\
\\
\Cmd{
SequenceA\\
   1 CAGCAGCACG CAAATTACCC ACTCCCGGCA CGGGAGGGTA GTGACGAAAA ATAACAATAC\\
  61 CCGTC\\
\\
SequenceB\\
   1 CAGGCACGCA AATTACCCAC TCCCGGCAGA GGTAGTGACA AAAAATAACG ATACGGGACT\\
  61 CCGTCAC\\
\\
SequenceC\\
   1 GGCACGGAGG TAGTGACGAA AAATAACGAT ACGGGACTCA TCCGAGGCCC CGTAATCGGA\\
\\
SequenceD\\
   1 AAATTACCCA CTCCCGGCAC GGAGGTAGTG ACGAAAAATA ACGATACGGG ACTCA\\
\\
SequenceE\\
   1 GAGGTAGTGA CGAAAAATAA CAATACAGGA CTCATATCCG AGGCCCTGTA ATT\\
}
\\
(Para proteinas, el asterisco "*" forza la interpretaci'on como proteina)\\
\Cmd{SequenceF\\
1 ILAVEELVI SLIVES\\
\\
SequenceG*\\
1 AAYVTTTCC KKYK}
\subsection{POY}
\noindent
\Pname{POY} y \Pname{Clustal}, utilizan el formato de FASTA (la primera l'inea tiene un 
\begin{math}
>
\end{math}
seguido por el nombre y comentarios de la secuencia; la siguiente l'inea comienza la secuencia como tal).\\
\\
\noindent
\Cmd{
> taxonA Comentarios\\
aaacgt\\
aac\\
> taxonB\\
aaacgt}

\section{Formatos de 'arboles}
\index{Formatos!\'Arboles}
\subsection{NONA}
\noindent
\Pname{NONA}, \Pname{Hennig86}, \Pname{WinClada} y \Pname{TNT} usan este formato; con \Cmd{*} indican que hay mas 'arboles y con \Cmd{;} que es el 'ultimo 'arbol. N'otese el espacio para separar los terminales. El primer taxon es 0. Al igual que en las matrices, \Cmd{p-} o \Cmd{p/} indican el final de lectura del archivo. Winclada puede incluir al incio la lista de nombres, pero no es compatible con otros programas.\\
\\
\noindent
\Cmd{
tread 'tres arboles'\\
(0 (1 (2 (3 4 ))))*\\
(0 (1 (2 3 4 )))*\\
(0 ((1 2 )(3 4 )));\\
p/;}\\
\\
\Cmd{
tread 'solo un arbol'\\
(0 (1 (2 (3 4 ))));\\
p/;}
\subsection{PAUP*}
En \Pname{PAUP*} el 'arbol est'a embebido en el archivo de la matriz o en un formato aparte. Los grupos son separados por comas y  el primer taxon es 1.\\
\\
\noindent
\Cmd{
\#nexus\\
\\
begin trees;\\
  translate\\
    1 lamda,\\
    2 alpha,\\
    3 beta,\\
    4 gamma,\\
    5 out\\
  ;\\
  tree *primero=(5,(1,(2,(3,4))));\\
  tree politomico=(5,(1,(2,3,4)));\\
  tree tercero=(5,((1,2),(3,4)));\\
end;}\\
\\
El orden no altera los 'arboles en los programas. As'i:\\
\Cmd{(0,(1,(2,(3,4))))}\\
es igual a\\
\Cmd{((1,((3,4),2)),0)}





%\printindex



\end{document}
