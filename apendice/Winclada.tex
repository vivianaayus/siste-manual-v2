\subsection{WinClada}
\noindent
Autor: Nixon, 2002.\\
Plataforma: Windows 9x o superior. [Funciona MUY bien en \Pname{wine} dentro de Linux].\\
Disponibilidad: Shareware (30 d\'ias de prueba), tiene un costo de US 50.00.\\
http://www.cladistics.com
% \href{http://www.cladistics.com}.
\\El programa require de \Pname{NONA} para realizar las b'usquedas (\Pname{PIWE}/\Pname{Hennig86} son deseables, pero no obigatorios).
\paragraph*{}
Bajo Windows, este es uno de los programas m'as 'utiles, tanto para principiantes como para iniciados. El programa tiene dos entornos para manejar matrices y 'arboles. El manejo e impresi'on de 'arboles permite que se tengan im'agenes listas para publicaci'on con pocos pasos y sin necesidad de retoque posterior. Tiene un par de ''conflictos'' con el manejo de im'agenes que pueden ser inc'omodos; adem'as, el usuario debe tener presente que el programa puede numerar (taxa, caracteres y 'arboles) desde cero (la manera l'ogica para los programadores), o desde uno, por lo que se debe tener cuidado al escribir el texto y referenciar la gr'afica.
\subparagraph*{WinDada}
\Gui{Matrix}: Le permite cambiar algunos aspectos de la matriz (crearlas, cambiar tama~no, salvarlas, fusionarlas).\\
\Gui{Edit}: Lo m'as importante es que permite que los datos sean o no modificables. Adem'as, permite adicionar polimorfismos.\\
\Gui{Terms/Chars}: Estos men'u le permiten modificar cosas espec'ificas de terminales y caracteres, tales como nombre y orden, seleccionarlos, borrarlos, adicionar.\\
\Gui{View}: Opciones de presentaci'on. Adem'as, puede ver estad'isticos de los caracteres, aditivos, activos, pasos m'inimos y m'aximos; y le permite intercambiar modo num'erico o IUPAC (para DNA).\\
\Gui{Output}: Le permite exportar en formatos diferentes, o informaci'on variada sobre los caracteres.\\
\Gui{Key}: Si tiene los caracteres nominados, funciona como clave interactiva.\\
\Gui{Analize}: Tiene diferentes entradas para b'usquedas y c'alculo de soporte. En \Gui{heuristics}, es posible configurar la cantidad m'axima de 'arboles a retener, la cantidad de 'arboles por cada r'eplica ('arbol de Wagner+TBR) (i.e. estrategias NONA/PAUP) y otros detalles de la b'usqueda, como especificar una semilla para los n'umeros aleatorios (0 es el tiempo interno de la computadora). En \Gui{ratchet}: se puede modificar la cantidad de iteraciones de ratchet, el n'umero de 'arboles a retener por iteraci'on y el n'umero de b'usquedas de ratchet, bien sean secuenciales o simult'aneas. Mientras que con \Gui{bootstrap/jackknife/CR with NONA} permite manipular las diferentes opciones para hacer soporte con remuestreo. Recuerde borrar todos los 'arboles antes de ejecutar la b'usqueda de remuestreo.
\subparagraph*{CPanel}
Para llenar la matriz.\\
\Gui{Mode}: Cambia el modo de acceso de los caracteres.
\subparagraph*{Winclados}
Para manejar el 'arbol. En general, la mayor parte de estas acciones son accesibles desde la barra de herramientas con el rat'on.\\
Las teclas F2-F12 tienen (con y sin SHIFT) diversas funcionalidades de visualizaci'on, como engrosar/adelgazar el ancho de las ramas, expandir/contraer la longitud, moverse entre 'arboles.\\
\Gui{Trees}: opciones para visualizar y manipular el 'arbol sin modificar su topolog'ia (forma, consensos, ir a un 'arbol determinado).\\
\Gui{Nodes}: Para ver frecuencias de nodos, desafortunadamente su funcionalidad es muy inestable, y a veces produce resultados inesperados.\\
\Gui{HashMarks}: Para ver las transformaciones en los nodos.\\
\Gui{Edit/Mouse modes:} Modifica las acciones posibles del rat'on para modificar el 'arbol, como mover nodos, colapsarlos, cambiar la ra'iz.\\
\Gui{Diagnoser}: Para mapear caracteres.
