% 
\renewcommand{\chaptername}{}
\chapter{Bibliograf'ia recomendada}
\noindent
Antezana M. 2003. When being ''most likely'' is not enough: examining the performance of three uses of the parametric bootstrap in phylogenetics. \textit{Molecular Evolution} 56:198-222\\
Bremer K. 1994. Branch support and tree stability. \textit{Cladistics} 10:295-304.\\
Camin J.H. \& Sokal R.R. 1965. A method for deducing branching sequence in phylogeny. \textit{Evolution} 19:311-326.\\
Carpenter, J.M. 1988. Choosing among multiple equally parsimonious cladograms. \textit{Cladistics} 4:291-296.\\
Coddington J. \& Scharff N. 1994. Problems with zero-length branches. \textit{Cladistics} 10:415-423.\\
Dayoff N.O. 1969. Computer analysis of protein evolution. \textit{Scientific American} 221:87-95.\\
deLaet J. 2005. Pseudocode for some tree search algorithms in phylogenetics. Publicado por el autor. Disponible en ineternet: http://www.plantsystematics.org/publications/jdelaet\\
Edgar R. C. 2004.  MUSCLE: multiple sequence alignment with high accuracy and high throughpu.\textit{Nucleic Acids Research} 32(5): 1792-97. 
Farris J.S. 1969. A successive approximations approach to character weighting. \textit{Systematic Zoology} 18: 374-385.\\
Farris J.S. 1970. Methods for computing Wagner trees. \textit{Systematic Zoology} 19:83-92.\\
Farris J.S. 1997. the future of phylogeny reconstruction. \textit{Zoologica Scripta} 26: 303+311.\\
Farris J.S. 2001. Support weighting. \textit{Cladistics} 17: 389-394.\\
Farris J.S., Kluge A.G. \& Eckardt M.J. 1970. On predictivity and efficiency. \textit{Systematic Zoology} 19:363-72.\\
Farris J.S., Albert V.A., Kallersjo M., Lipscomb D. \& Kluge A.G. 1996. Parsimony jackknifing outperforms neighbor-joining. \textit{Cladistics} 12:99-24.\\
Felsenstein J. 2004. \textit{Inferring phylogenies}. Sinauer.\\
Fitch W. M. 1971. Toward defining the course of evolution: minimum change for a specific tree topology. \textit{Systematic Zoology} 20:406-416.\\
Frost D., Janies D., Mouton P.L.F.N. \& Titus, T. 2001. A molecular perspective on the phylogeny of the Girdled lizards (Cordylidae, Squamata). \textit{American Museum Novitates} 3310: 1-10.\\
Grant T. \& Kluge A.G. 2004. Transformation series as an ideographic character concept. \textit{Cladistics} 20:23-31.\\
Giribet G. \& Wheeler W. 1999. On gaps. \textit{Molecular phylogenetics and evolution} 13:132-143.\\
Giribet G., Wheeler W. \& Mouna J. 2002. DNA multiple sequence alignments. En: \textit{Molecular systematics and evolution: theory and practice} (Ed. R. DeSalle, G. Giribet, W. Wheeler). Birkhauser, pp. 107-114.\\ 
Goloboff P.A. 1993. Estimating character weights during tree search. \textit{Cladistics} 9: 83-91.\\
Goloboff P.A. 1995. Parsimony and weighting: a reply to Turner and Zandee. \textit{Cladistics} 11: 91-104.\\
Goloboff P.A. 1997. Self-weighted optimization: Tree searches and state reconstructions under implied transformation costs. \textit{Cladistics} 13:225-245.\\
Goloboff P.A. 1998. Tree searches under Sankoff parsimony. \textit{Cladistics} 14:229-237.\\
Goloboff P.A. 1999. Analyzing large data sets in reasonable times: solutions for composite optima. \textit{Cladistics} 15:415-428.\\
Goloboff P.A., Farris J.S., Kallersjo M., Oxelman B., Ram'irez M.J. \& Szumik C.A. 2003. Improvements to resampling measures of group support. \textit{Cladistics} 19:324-332.\\
Goloboff P.A. \& Farris J.S. 2001. Methods for quick consensus estimation. \textit{Cladistics} 17:S26-S34.\\
Goloboff P.A. \& Pol, D. 2005. Parsimony and Bayesian phylogenetics. En: \textit{Parsimony, Phylogeny, and Genomics} (Ed. V.A. Albert). Oxford, pp. 148-162.\\
Goloboff P.A., Mattoni C.I. \& Quinteros A.S. 2006. Continous characters analyzed as such. \textit{Cladistics} 22:589-601.\\
Hennig W. 1968. \textit{Fundamentos de una sistemtica filogen'etica}. Eudeba.\\
Huelsenbeck J.P., Hillis D.M. \& Jones R. 1996. Parametric bootstrapping in molecular phylogenetics: applications and performance. En: \textit{Molecular zoology: advances, strategies, and protocols} (Ed. J.D. Ferraris, S.R. Palumbi). Wiley-Liss, pp. 19-45.\\  
Huelsenbeck J.P., Larget B., Miller R.E. \& Ronquist F. 2002. Potential applications and pitfalls of Bayesian inference of phylogeny. \textit{Systematic Biology} 51:673-688.\\
Huelsenbeck J.P., Ronquist F., Nielsen R. \& Bolback J.P. 2001. Bayesian inference of phylogeny and its impact on evolutionary biology. \textit{Science} 294: 2310-2314.\\
Jukes T.H. \& Cantor C.R. 1969. Evolution in protein  molecules. En: \textit{Mammalian protein metabolism}, vol. 3 (Ed. M.N. Munro). Academic, pp. 21-132.\\
Kearney M. 2002. Fragmentary taxa, missing data, and ambiguity: mistaken assumptions and conclusions. \textit{Systematic Biology} 51:369-381.\\
Kjer K.M., Blahnik R.J. \& Holzenthal R.W. 2001. Phylogeny of Trichoptera (caddisflies): characterization of signal and noise within multiple datasets. \textit{Systematic Biology} 50: 781-816.\\
Kjer K.M., Blahnik R.J. \& Holzenthal R.W. 2002. Phylogeny of caddisflies (Insecta, Trichoptera). textit{Zoologica scripta} 31: 83-91.\\
Kluge A.G. 1997. Sophisticated falsification and research cycles: consequences for differential character weighting in phylogenetic systematics. \textit {Zoologica scripta} 26: 349-360.\\
Kluge A.G. 2002. Distinguishing -or- from -and- and the case for historical identification. \textit{Cladistics} 18:585-593.\\
Kluge A.G. 2003. The repugnant and the mature in phylogenetic inference: atemporal similarity and historical identity. \textit{Cladistics} 19:356-368.\\ 
Kluge A.G. \& Farris J.S. 1969. Quantitative phyletics and the evolution of anurans. \textit{Systematic Zoology} 18:1-32.\\
Lewis P.O. 2001. A likelihood approach to estimating phylogeny from discrete morphological data. \textit{Systematic Biology} 50: 913-925.\\
Maddison W.P. \& Maddison D.R. 1992. \textit{MacClade version 3: Analysis of phylogeny and character evolution}. Sinauer.\\
Maddison D.R., Swofford D.L. \& Maddison W.P. 1997. NEXUS: An extensible file format for systematic information. \textit{Systematic Biology}, 46:590-621.\\
Miranda-Esquivel D.R., Garz'on I. \& Arias J.S. 2004. ?`Taxonom'ia sin historia? Entom'ologo 32, 4-7.\\
Miyamoto M.M. 18�985. Consensus cladograms and general classifications. \textit{Cladistics} 1: 186-189.\\
Neff N.A. 1986. A rational basis for a priori character weigthing. \textit{Systematic Zoology} 35:110-123.\\
Nelson G. 1979. Cladistic analysis and synthesis: principles and definitions, with a historical note on Adanson- Familles des plants (1763-1764). \textit{Systematic Zoology} 28:1-21.\\
Nixon K.C. 1999. The parsimony ratchet, a new method for rapid parsimony analysis. \textit{Cladistics} 15:407-414.\\
Nixon K.C. \& Carpenter J.M. 1996. On consensus, collapsibility, and clade concordance. \textit{Cladistics} 12:305-321.\\
Lewis P. 2001. A likelihood approach to estimating phylogeny from discrete morphological character data. \textit{Systematic Biology} 50:913-925.\\
Page R.D. \& Holmes E.C. 1998. \textit{Molecular Evolution: A Phylogenetic Approach}. Blackwell Publishers.\\
Patterson C. 1981. Significance of fossils in determining evolutionary relationships. \textit{Annual Review of Ecology and Systematics} 12:195-223.\\
Patterson C. 1982. Morphological character and Homology. En: \textit{Problems of phylogenetic reconstruction} (Ed. K.A. Joysey, A.E. Friday) . Academic, pp. 21-74.\\
de Pinna M.C.C. 1991. Concepts and tests of homology in the cladistic paradigm. \textit{Cladistics} 7:367-394.\\
Platnick N.I. 1979. Philosophy and the transformation of cladistics. \textit{Systematic Zoology} 28: 537-546.\\
Pleijel F. 1995. On character coding for phylogeny reconstruction. \textit{Cladistics} 11:309-315.\\
Posada D. \& Crandall K.A. 2001. Selecting the best-fit model of nucleotide substitution. \textit{Systematic Biology} 50:580-601.\\
Quicke D.L.J, Taylor J. \& Purvis A. 2001 Changing the landscape: a new strategy for estimating large phylogenies. \textit{Systematic Biology} 50:60-66.\\
Ram'irez M.J. 2005. Resampling measures of group support: a reply to Grant and Kluge. \textit{Cladistics} 21:83-89.\\
Rice K.A., Donoghue M.J. \& Olmstead R.G. 1997. Analyzing large data sets: rbcL 500 revisited. \textit{Systematic Biology} 46:554-563.\\
Richards R. 2002. Kuhnian valures and cladistic parsimony. \textit{Perspectives on Science} 10:1-27.\\
Richards R. 2003. Character individuation in phylogenetic inference. \textit{Philosophy of Science} 70:264-279.\\
Rieppel O. \& Kearny M. 2002. Similarity. \textit{Biological Journal of the Linnean Society} 75:59-82.\\ 
Scotland R. W., Olmstead R.G. \& Bennett J.R. 2003. Phylogeny reconstruction: the role of morphology. \textit{Systematic Biology} 52:539-548.\\
Sharkey M.J. \& Leathers J.W. 2001. Majority does not rule: the trouble with majority-rule consensus trees. \textit{Cladistics} 17:282-284.\\
Sokal R.R. 1983. A phylogenetic analysis of the Caminalcules. I. The data base. \textit{Systematic Zoology} 32(2): 159-184.\\
Steel M. 2002. Some statistical aspects of the maximum parsimony method. En: \textit{Molecular Systematics and Evolution: Theory and Practice} (Ed. R. DeSalle, G. Giribet \& W. Wheeler). Birkhauser, pp. 125-140.\\
Strong, E.E. \& Lipscomb, D. 1999. Character codding and inaplicable data. \textit{Cladistics} 15:363-371.\\
Swofford D.L. 1991. When are phylogeny estimates from molecular and morphological data incongruent? En: \textit{Phylogenetic Analysis of DNA Sequences} (Ed. M.M. Miyamoto, J. Cracraft). Oxford, pp. 295-333.\\
Swofford D.L. Olsen G.J., Waddell P.J. \& Hillis D.M. 1996. Phylogenetic inference. En: \textit{Molecular systematics} (Ed. D.M. Hillis, C. Moritz, B.K. Mable). Sinauer, pp. 407-514.\\
Tuffley C. \& Steel M. 1997. Links between maximum likelihood and maximum parsimony under a simple model of site substitutions. \textit{Bull. Math. Biol}. 59, 581-607.\\
Wagner W.H. 1961. Problems in classification of ferns. En: \textit{Recent advances in botany}, vol. 1. Toronto, pp. 841-844.\\
Wheeler W. 1995. Sequence alignment, parameter sensibility, and the phylogenetic analysis of molecular data. \textit{Systematic biology} 44:321-331.\\
Wheeler W. 1996. Optimization alignment: the end of multiple sequence alignment in phylogenetics? \textit{Cladistics} 12: 1-9.\\
Wheeler W. 1999. Fixed character states and the optimization of molecular sequence data. \textit{Cladistics} 15: 379-386.\\
Wheeler W., Arango C.P., Grant T., Janies D., Var'on A., Aagesen L., Faivovich J., DHaese C., Smith W.L. \& Giribet G. 2006. textit{Dynamic homology and phylogenetic systematics: a unified approach using POY}. American museaum of natural histoy.\\
Wiley E.O. 1981. \textit{Phylogenetics}. John Wiley and Sons.\\
