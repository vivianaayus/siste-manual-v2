\subsection{PAUP*}
\noindent
Autor: Swofford, 2002.\\
Plataforma: MWindows, MacOS, Unix.\\
Disponibilidad: el programa para l interface gr\'afica tiene un costo entre US 80.00 a US150.00 dependiendo del sistema operativo y es distribuido por Sinauer, mientras que para l\'inea de comandos es gratuito.\\
http://www.sinauer.com/detail.php?id=8060.


\paragraph*{}
\Pname{PAUP*} es uno de los programas m\'as usados en cuanto a b\'usquedas mediante ML, o si el sistema operativo es Mac. Dados los costos es muy posible que la interface dominante llegue a ser por l\'inea de comandos; la interface para Windows y MacOSX en modo gr\'afico tiene el mismo acercamiento; como \Pname{NONA} y \Pname{TNT} tiene la opci\'on de manejar un gran n\'umero de parametros que permiten hacer una b\'usqueda \textbf{sobre pedido}, adicionalmente realiza todos los consensos directamente sin necesitar un segundo programa. Su gran desventaja es su velocidad y la falta de un lenguaje de macros.


\cmd{set}:esta funci\'on define la configuraci\'on b\'asica de algunos par\'ametros. Son importantes \cmd{increase=no}, para que no le pregunte si desea aumentar el n\'umero de \'arboles, y \cmd{maxtrees=N} para dar como opci\'on que guarde un n\'umero de \'arboles (N).\\
\cmd{exec}: para abrir la matriz de datos o archivos de instrucciones.\\
\cmd{hsearch}: es la funci\'on de b\'usqueda de cladogramas. Las opciones de este comando son \cmd{addseq=random}, lo que asegura que la entrada de datos para el \'arbol de Wagner es al azar.\\
\cmd{swap=tbr/spr}: es el tipo de permutaci\'on.\\
\cmd{nreps=X}: indica el n\'umero de r\'eplicas que se har\'a en la b\'usqueda.\\ 
\cmd{nchuck=X ckuckscore=1}: le permite usar la estrategia NONA, indicando c\'uantos \'arboles desea por r\'eplica. Si usted desea hacer solo permutaci\'on de ramas del \'arbol previamente construido, la serie de instrucciones ser\'ia \cmd{search start=current chuckscore=no}.\\
Para guardar los \'arboles utilice el comando \cmd{savetre}.\\
Con el comando \cmd{contree} se generan los \'arboles de consenso. Estos deben ser guardados cuando se ejecuta el comado (par\'ametro \cmd{treefile=nombre}, donde nombre es el achivo.). Los m\'etodos de medici\'on de soporte implementados son el \textit{bootstrap} con el comando \cmd{bootsptrap} y el \textit{jackknife} con \cmd{jackknife}.\\
En caso de dudas, el programa posee una ayuda en linea, cons\'ultela usando \cmd{help} o escribiendo \cmd{help comando} para un comando espec\'ifico; para ver las opciones del comando use \cmd{comando ?}.