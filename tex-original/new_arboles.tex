\chapter{'Arboles}
\section*{Introducci'on}
\thispagestyle{empty}
\large{\textsf{Una vez realizado un an\'alisis filogen\'etico,  %(relaciones creadas por los caracteres)
el resultado es un agrupamiento que tradicionalmente se dibuja como un \'arbol donde los taxa usados estan como \emph{hojas} o \emph{terminales} del \'arbol. Las diferentes ramas se unen en \emph{nodos} o \emph{componentes}. As\'i, el agrupamiento ((Terminal 1, Terminal 2), Terminal 3) puede representarse con el \'arbol:}}\vspace*{0.7cm}
 \begin{center}
%
% http://tex.stackexchange.com/questions/2340/how-to-make-a-3-level-deep-tree-with-tikz
%
\begin{tikzpicture}[level distance=3cm,
  level 1/.style={sibling distance=9cm},
  level 2/.style={sibling distance=6cm}]
  \node {Ra\'iz}
    child {node {Terminal 3}}
    child {node {Nodo interno}
    child {node {Terminal 1 = nodo (hoja)}}
      child {node {Terminal 2}}
    };

\end{tikzpicture}


\end{center}\vspace*{0.7cm}

\textsf{\large{Es importante diferenciar entre el \textbf{contenido} 
del nodo, que son los terminales que forman el nodo (el nodo 1 tiene 
dos terminales: 1 y 2), y la \textbf{informaci\'on} del nodo, que es 
c\'omo est\'an organizados estos terminales (Nelson, 1979). En general, 
para muchos autores un componente representa un grupo monofil\'etico 
(i.e., definici\'on \emph{topol\'ogica} de monofilia).\\ En clad\'istica, 
los \'arboles son tambi\'en conocidos como \textbf{cladogramas} y el 
t\'ermino es usado sin distinci\'on. En clad\'istica (pero no en otras 
metodolog\'ias) los cladogramas deben tener los nodos soportados por 
transformaciones de caracteres. En caso de no tener 
transformaciones, el nodo no est\'a soportado o es un nodo de longitud 
cero, es decir, es un artefacto de los programas usados para obtener 
los \'arboles (v\'ease Coddington \& Scharff, 1994; Goloboff, 1998). 
El n\'umero de transformaciones es un estad\'istico importante que hay 
que tener en cuenta, en la mayor\'ia de los casos se lo usa como 
criterio para seleccionar entre diferentes cladogramas, se escoge el 
que sugiera el  menor n\'umero de transformaciones. En los m\'etodos 
estad\'isticos generalmente es importante la longitud de las ramas. 
Esta suele representarse como una probabilidad o frecuencia de 
transformaciones y suelen presentarse los \'arboles con una escala 
que muestra la longitud. S\'olo la dimensi\'on que va de las ramas a los 
terminales tiene importancia en ese caso, el ``ancho'' del \'arbol 
s\'olo es para acomodar los diferentes terminales usados. En los 
cladogramas, las dimensiones no tienen ning\'un significado 
especial.}}

\section{T\'ecnicas}

\textsf{\large{Existen diferentes programas para manipular y generar 
\'arboles; la mayor\'ia de ellos permiten interacci\'on 
matriz-\'arbol, mientras que otros solo utilizan \'arboles. En general, los 
programas de an\'alisis filogen\'etico permiten manipular \'arboles 
con una presentaci\'on gr\'afica rudimentaria. Algunos programas 
permiten guardar informaci\'on adicional a la topolog\'ia en el 
mismo \'arbol como la longitud de la rama que conduce al nodo o 
terminal o una etiqueta. Los programas que pueden trabajar 
independientemente de la matriz suelen estar dise\~nados para la 
impresi\'on o exportaci\'on gr\'afica de los \'arboles, soportando 
cambios de topolog\'ia, longitudes y etiquetas de las ramas, pero no 
transformaciones de caracteres.}}

\textsf{\large{Una forma de expresar los \'arboles en formato de 
texto (por ejemplo para usar con programas o para el resumen de un 
art\'iculo) es la notaci\'on de par\'entesis, donde los par\'entesis 
limitan los nodos, dependiendo de los autores o los programas, los 
grupos hermanos son separados por espacios, comas o s\'imbolos de 
suma; por ejemplo (a (b c)), es equivalente a (a+(b+c)) y a (a, (b, 
c)). Finalmente tenga en cuenta que en los \'arboles filogen\'eticos 
el orden de los terminales, sin cambiar la topolog\'ia, no altera el 
contenido del \'arbol, por ejemplo (a (b c)) es exactamente igual a 
((c b) a) y a ((b c) a)}}.


\section{Materiales}

\textsf{\large{Matriz de Vertebrados (vertebrados.dat).\\
Matriz de datos con \'arboles (datos.arbol.dat).\\
\'arbol simple (datos.tre).\\
\'arbol con longitud de ramas (datos.long.tre).\\
Software: Editor de Textos, \Pname{WinClada}, \Pname{NONA}, \Pname{Mesquite}, \Pname{Figtree\footnote{http://tree.bio.ed.ac.uk/software/figtree/}}, \Pname{TNT}.}}
\textsf{\large{
\section{M\'etodos}
\begin{enumerate}
\item Desde \Pname{WinClada+NONA}, \Pname{Mesquite} o \Pname{TNT}
\begin{enumerate}


\item Desde \Pname{WinClada} o \Pname {TNT} (para Windows), haga una 
b'usqueda de cladogramas para su matriz de vertebrados, usando los 
par'ametros por omisi'on (men'u \Gui{Analize} / \Gui{heuristics}, 
\Gui{traditional search}).

\item \textbf{En \Pname{Mesquite},  
\Pname{MacClade} o \Pname{WinClada}} abra el archivo ''datos.arbol.dat''. Indique la cantidad de 'arboles 
obtenidos y la longitud de los mismos.

\item Examine las agrupaciones obtenidas, identifique los 
 caracteres que soportan los nodos.

\item   Mapee los caracteres en 
los \'arboles usando distintos tipos de optimizaciones: en \Pname
{WinClada} ACTRAN, DELTRAN y no ambigua, En TNT (para Windows) 
(men'u \Gui{Optimize / character / reconstructions} y selecciones un 
solo arbol y algunos caracteres) y en \Pname{Mesquite}: Dollo e 
Irreversible (C-S); revise c'omo cambian las optimizaciones en los 
distintos nodos.\index{Caracter!Optimizaci\'on} 



\item Haga una b\'usqueda de cladogramas para su matriz de vertebrados, usando los par\'ametros por omisi\'on.
\item Reporte, la cantidad de 'arboles obtenidos y la longitud de los mismos.
\item Examine las agrupaciones obtenidas, identifique los caracteres que soportan los nodos.
\item Mapee los caracteres en los \'arboles usando distintos tipos de optimizaciones: en \Pname{WinClada} ACTRAN, DELTRAN y no ambigua, En \Pname{TNT} (para Windows) (\Gui{Optimize $>$ Character $>$ Reconstructions} y seleccione un solo \'arbol y algunos caracteres) y en \Pname{Mesquite}: Dollo e Irreversible (C-S); revise c'omo cambian las optimizaciones en los distintos nodos.
\item Cambie de posici\'on algunos nodos o terminales de la topolog'ia, y observe c'omo cambia la longitud del cladograma y los estados asignados a los nodos o la forma como los caracteres son mapeados, pruebe tambi\'en moviendo ramas completas.
\end{enumerate}
\item Tanto en \Pname{Figtree} (usando solo el \'arbol) como en \Pname{WinClada}, \Pname{TNT} (para Windows), o en \Pname{Mesquite} (usando el archivo con la matriz incluida):
\begin{enumerate}
\item Cambie la ra'iz del 'arbol.
\item Cambie de posici'on nodos y terminales.
\item Colapse alg'un nodo.
\end{enumerate}
\item En \Pname{Mesquite} o \Pname{Figtree}:
\begin{enumerate}
\item Cambie el orden de los terminales sin cambiar la topolog'ia (las relaciones entre terminales) del 'arbol.
\item Coloque etiquetas en los nodos, y salve el 'arbol.
\item Revise ese archivo en un editor de texto para determinar c'omo se colocaron tanto las etiquetas como las longitudes.
\item Abra el 'arbol con longitud de ramas.
\end{enumerate}
\end{enumerate}}}
\subsection{Programas}
\textsf{\large{Dependiendo de la plataforma que trabaje, usted tiene disponible distintos programas. \Pname{Mesquite} y \Pname{Figtree} son v'alidos para todas las plataformas (son gratuitos). \Pname{Mesquite} es muy lento si su m'aquina es lenta (requiere de la m'aquina virtual de Java). Sobre Windows usted cuenta con \Pname{WinClada} (requiere la matriz de datos), su interfaz de impresi'on requiere mucha pr'actica (no es intuitiva). Otras alternativa, si usa Linux, pueden ser \Pname{NJPlot o GNUplot}. Para la impresi'on final del 'arbol para publicaci'on, una opci'on puede ser \Pname{Treegraph\footnote{http://www.math.uni-bonn.de/people/jmueller/extra/treegraph} o R\footnote{http://cran.r-project.org}}, que aunque poco intuitivo esta disponible para todas las plataformas.}}
\textsf{\large{
\subsection{Comandos}
En general, la manipulaci'on de arboles se activa con un men'u (en \Pname{WinClada} \Gui{Edit/Mouse}, en \Pname{FigTree} directamente como barra de herramientas en la ventana, en \Pname{Mesquite} se usa el men\'u de herramientas de la ventana de edici'on de \Pname{TreeView}). Las acciones se realizan al seleccionar con el rat\'on la rama (\Pname{WinClada}) o arrastrando las ramas (para moverlas en \Pname{Figtree}). M'as que comandos, en esta pr\'actica es importante manipular el rat'on.}}
\textsf{\large{
\section{Preguntas}
\subsection{Pr'actica}
\begin{enumerate}
\item Examine sus caracteres y cambie su aditividad-no aditividad. ¿Cambia esto la forma como mapean? Haga este ejercicio para varios caracteres, binarios y multiestado.
\item Si cambia la ra\'iz, ¿cambia la longitud? ¿Por qu\'e cree que se presenta el resultado que obtuvo? ¿Es (y por qu\'e) una ventaja o una desventaja?
\item Dibuje el siguiente cladograma:} 
\small{
\emph{(Lampreas (Tiburones (Esturi'on Tele'osteos) (Celacanto (Peces-pulmonados (Anfibios (Mam'iferos (Tortugas (Lagartos (Cocodrilos Aves)))))))))}
}
\large{
\item Si la ra'iz esta colocada entre Aves y Cocodrilos, ¿c'omo es la topolog'ia resultante?
\item Convierta la topolog'ia dibujada en la pregunta No. 4 a notaci'on de par'entesis.
\end{enumerate}
\subsection{Generales}
\noindent
Aparte de no tener un an'alisis expl\'icito, existe una gran diferencia entre los 'arboles filogen'eticos actuales y sus ''equivalentes'' usados por algunos tax\'onomos (por ejemplo, Haeckel, Romer, etc.) ?`Es usted capaz de descubrirla? \textbf{Clave}: intente dibujar alguno de esos 'arboles al estilo actual.}}
\section{Literatura recomendada}
\noindent
\textsf{\large{Page \& Holmes, 1998 [El cap'itulo 2 esta dedicado a los 'arboles y presenta muy buenas ilustraciones. Tambi\'en puede consultar la p\'agina de docencia de Rod Page http://taxonomy.zoology.gla.ac.uk/teaching/index.html]}}.
